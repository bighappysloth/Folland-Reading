\documentclass[../main-v2-manifolds.tex]{subfiles}
\begin{document}
\fchapter{1: Manifolds}
% Introduce terminology. linear, toplinear, morphism, isomorphism. And Laut, List, etc. How List is open in the space of toplinear endomorphisms on a space $E$.

\topheader{Introduction}
We begin with some terminology concerning maps between Banach spaces. Let $E$ and $F$ be Banach spaces over $\real$.
\begin{itemize}
    \item $\mathcal{L}(E,F)$, (resp. $L(E,F)$) = Linear (resp. toplinear) maps between $E$ and $F$, 
    \item $\Topliso(E,F)$ = toplinear isomorphisms between $E$ and $F$,
    \item $\Laut(E)$ = toplinear automorphisms on $E$
\end{itemize}
\begin{remark}[Laut is open in L(E,E)]
    The space of toplinear automorphisms is open in the strong topology in the space of toplinear endomorphisms.
\end{remark}

\topheader{The structure of a manifold}
It is fruitful to \emph{construct} the manifold rather than \emph{define} it. We also insist on working with open sets of Banach spaces instead coordinate functions as our primary data.\\

We will be working in the category of $C^p$ Banach spaces (all Banach spaces are assumed to be over $\real$). Its morphisms are $C^p$ morphisms: the maps which are continuously $p$-times differentiable (but not necessarily linear). Note that if $p\geq 0$, every toplinear morphism is a $C^p$ morphism, and every toplinear isomorphism is a $C^p$ isomorphism. However, a bijective $C^p$ morphism is usually not a $C^p$ isomorphism. 

\begin{definition}[Chart]\label{def:chart}
    Let $X$ be a non-empty set. A \emph{chart on $X$ modelled on a Banach space $E$} is a tuple $(U,\varphi)$, such that $U\subseteq X$,  $\varphi(U)=\hat{U}$ is an \emph{open} subset of $E$, and $\varphi$ is a bijection onto $\hat{U}$.
\end{definition}
\begin{definition}[Compatibility]\label{def:compatibility}
    Let $(U,\varphi)$ and $(V,\psi)$ be charts on $X$ modelled on $E$, they are called $C^p$ compatible if $U\cap V=\varnothing$, or 
    \begin{itemize}
        \item $\varphi(U\cap V)$ and $\psi(U\cap V)$ are \emph{both} open subsets of $E$, and
        \item the \emph{transition map} $\psi\circ\varphi^{-1}: \varphi(U\cap V)\to \psi(U\cap V)$ is a $C^p$ isomorphism between open subsets of $E$.
    \end{itemize}
    It should be clear that compatibility is an equivalence relation on the space of charts of $X$ (that are modelled on $E$). 
\end{definition}
\begin{definition}[Atlas]\label{def:atlas}
    Let $X$ be a non-empty set. A \emph{$C^p$ atlas on $X$ modelled on $E$} is a pairwise $C^p$ compatible collection of charts $\{(U_\alpha,\varphi_\alpha)\}$ whose union over the domains cover $X$.
\end{definition}
\begin{remark}[Omissions]
    If the \emph{model space} \( E \) is implied, we will not explicitly reference it. When operating 'within category', we might refer to two charts as \emph{compatible} or \emph{smoothly compatible}, implying they are $C^p$ compatible. This comes from the perspective that, in the context of $C^p$ manifolds, any smoothness exceeding $C^p$ is deemed sufficiently smooth for our purposes.
\end{remark}

Let $X$ be a non-empty set, equipped with a $C^p$ atlas $\{(U_\alpha,\varphi_\alpha)\}$ modelled on $E$. If $\alpha$ and $\beta$ both index the atlas, we write  $U_{\alpha\beta} = U_{\alpha}\cap U_{\beta}$.\\

Suppose $U_{\alpha\beta}$ is non-empty. Then, (by definition) the images $\varphi_{\alpha}(U_{\alpha\beta})$, $\varphi_{\beta}(U_{\alpha\beta})$ are \emph{both} open subsets of $E$, and we will denote the transition map by

\begin{equation}\label{def-transition-map}
    \varphi_{\beta}\circ\varphi_{\alpha}^{-1}=\varphi_{\beta\alpha^{-1}}: \varphi_{\alpha}(U_{\alpha\beta})\to\varphi_{\beta}(U_{\alpha\beta})
\end{equation}

If $p\in (U,\varphi)$, we write $\hat{p}$ for $\varphi(p)$ if there is no room for ambiguity. From \Cref{def:compatibility,def:atlas}, the compatibility relation on charts descends into a compatibility relation on the space of atlases, whose properties are summarized in the following note.

\begin{note}[Descent of an equivalence relation]\label{note:equivalence-relation-descends}
    Let \( \Omega \) be a non-empty set with an associated equivalence relation \( \sim \). This relation \( \sim \) induces another equivalence relation on the set containing all subsets of equivalence classes from \( \Omega \). Suppose \( A \) and \( B \) are subsets of the equivalence classes \( [A] \) and \( [B] \) respectively. The condition \( A \sim B \) holds if and only if for all elements \( x \) in \( A \) and \( y \) in \( B \), \( x \sim y \).\\
    
     This is equivalent to stating that the union \( A \cup B \) lies entirely within some equivalence class, and further, that \( [A] \sim [B] \). The class \( [A] \) represents the largest subset of \( \Omega \) that is entirely contained within a single equivalence class (namely $[A]$ itself) and contains $A$ as a subset.
\end{note}

\begin{definition}[Structure determined by an atlas]\label{def:structure-of-manifold}
    The maximal atlas that contains $\mathcal{A}$ as a subset is called the \emph{$C^p$ structure determined by $\mathcal{A}$}. This maximal atlas is unique, by \cref{note:equivalence-relation-descends}.
\end{definition}
\begin{definition}[Manifold]\label{def:manifold}
    A \emph{$C^p$ manifold modelled on $E$} is a non-empty set $X$ with a $C^p$ structure modelled on $E$. We sometimes refer to the manifold as the smooth structure, rather than the set $X$ itself. $\operatorname{Man}^p$ refers to the \emph{category of $C^p$ manifolds}.
\end{definition}
\begin{wts}[$E$ is a manifold]\label{prop:banach-space-is-manifold}
    Let $p\geq 1$. The identity map $\id{E}: E\to E$ defines an atlas on $E$, which determines a structure called the \emph{standard $C^p$ structure on $E$} or \emph{standard structure on $E$} if the class of morphisms is understood. We will call $(E,\id{E})$ the \emph{standard chart}, or the \emph{global chart} on $E$.
\end{wts}

\begin{wts}[Topology is unique on a manifold]\label{def:topology-unique-on-manifold}
    Let $X$ be a manifold modelled on $E$, it has a unique topology such that the domain for each chart in its smooth structure is open, and each chart is a homeomorphism onto its range (with respect to the subspace topology of $E$).
\end{wts}
\begin{proof}
    We offer a sketch of the proof. Fix a chart $(U,\varphi)$, it is clear that $U$ has to be in the topology of $X$, and because $\varphi: U\to \hat{U}$ is required to be a homeomorphism, we duplicate all the open sets in $\hat{U}$ by using the inverse image through $\varphi$. The collection of all such inverse images form a sub-basis, thus defines a unique topology as is well known.\\

    There is an alternate way of thinking about this 'induced topology'. Given a chart domain, there exists a unique coarsest topology such that all charts with the same chart domain are homeomorphisms onto their images. We can stitch these weak topologies together to form a ambient topology on $X$, as the chart domains cover $X$.
\end{proof}
\begin{remark}[Not necessarily Hausdorff]
    The topology generated is not necessarily Hausdorff, nor second countable. So $X$ may not admit partitions of unity, but for our current purposes we will work with this general definition. Because of the uniqueness of the topology, we sometimes refer to the topology as being part of the \emph{structure} of the manifold.
\end{remark}
\begin{wts}[Open subsets of manifolds]\label{prop:open-subsets-of-manifolds}
    If $U$ is an open subset of a $C^p$ manifold $X$, then $U$ is a $C^p$ manifold whose structure is determined by the atlas
    \begin{equation}\label{eq:open-subset-atlas}
    \bigset{(V,\varphi) \text{ in the structure of }X,\:\text{where } V\subseteq U}
    \end{equation}
\end{wts}
\begin{proof}
    The smooth structure of $X$ includes all possible restrictions to open sets; hence the set in \cref{eq:open-subset-atlas} defines an atlas, and a unique structure by \cref{def:structure-of-manifold}.
\end{proof}
\topheader{Morphisms between manifolds}
\begin{definition}[$C^p$ morphisms between manifolds]\label{def:smoothness}
Let $X$ and $Y$ be $C^p$ manifolds over the spaces $E$ and $F$. A map $F: X\to Y$ is a morphism in $\operatorname{Man}^p$ if for every $p\in X$, there exists charts $(U,\varphi)$ in $X$ and $(V,\psi)$ in $Y$ such that the image $F(U)$ is contained in $V$, and the conjugation of $F$ with respect to the two charts is $C^p$ smooth between open subsets of Banach spaces.

\begin{equation}\label{eq:smoothness}
    F_{U,V}\defined \psi F \varphi^{-1} \in C^p(\hat{U},\hat{V})    
\end{equation}

The map defined in \cref{eq:smoothness} is called the \emph{coordinate representation of $F$} with respect to the charts $(U,\varphi), (V,\psi)$. 
\end{definition}

\begin{remark}[Identifying charts with their domains]
The scenario in \cref{eq:smoothness} occurs so often that we decide to simply write 
\begin{equation}\label{eq:suitably-chosen-charts-1}
    F_{U,V} = \psi F\varphi^{-1}
\end{equation}
to mean there exists charts $(U,\varphi)$, $(V\psi)$ in the structure of $X$, $Y$ such that 
\begin{equation}\label{eq:suitably-chosen-charts-2}
    F(U)\subseteq V
\end{equation}
Consistent with our notation for identifying a chart by its chart domain, we write 
\begin{equation}\label{eq:suitably-chosen-charts-3}
    F_{U,V}(\hat{p}) = (\psi F\varphi^{-1})(\hat{p})
\end{equation}
for any morphism $F\in \Mor(X,Y)$, charts that satisfy \cref{eq:suitably-chosen-charts-2}. We refer to the map in \cref{eq:suitably-chosen-charts-3} as a \emph{coordinate representation of $F$ about $p$}.
\end{remark}

% Motivation for the definition of smoothness
\Cref{def:smoothness} may leave one unsatisfied. A common question that comes to mind is: why do we require the image $F(U)$ be contained in another chart domain in $Y$? There are two reasons.
\begin{enumerate}
    \item First, it is easily verified that the $C^p$ maps between open subsets of Banach spaces satisfy the usual functoral properties in its category. The definition of smoothness between Banach spaces is a purely local one, and it is defined between open subsets; and recall: \textbf{every chart domain $U$ in a manifold $X$ corresponds to an open subset $\hat{U}\subseteq E$ in the model space}. The necessity that $F(U)$ must be contained in a single chart domain of $Y$ is a relic of the original definition.
    \item Second, suppose $f$ is a map between $E$ and $F$, and the restriction of $f$ onto a family of open subsets $U_{\alpha}\subseteq E$ is $C^p$ for $p\geq 0$. If $\{U_\alpha\}$ is an open cover for $E$, then $f$ is continuous. \Cref{prop:smoothness-implies-cont-functorality} shows this equally holds for manifolds.
\end{enumerate}

%

\begin{wts}\label{prop:smoothness-implies-cont-functorality}
    Every $C^p$ morphism between manifolds is a continuous map, and the composition of $C^p$ morphisms is again a morphism.
\end{wts}
\begin{proof}
    The first claim follows immediately from \cref{eq:smoothness}, since $p$ is arbitrary, choose any neighbourhood $W$ of $F(p)$, by shrinking this neighbourhood, it suffices to assume it is a subset of the chart domain $V$. The charts on $X$ and $Y$ are homeomoprhisms, and unwinding the formula shows that $F\vert_{U} = \psi^{-1}F_{U,V}\varphi$, so that
    \[
        U\cap F^{-1}(W) = (F\vert_U)^{-1}(W)\quad\text{is open in }X
    \]
    To prove the second, let $\UL{X}[3]$ be manifolds modelled over $\UL{E}[3]$, and $F_1$, $F_2$ is smooth between $X_i$ such that $F_2\circ F_1$ makes sense. Since $F_1$ is smooth, there a pair of charts $(U_i,\varphi_i)\in X_i$ for $i = 1,2$ about each $p\in X_1$ such that $F_1{_{U_1,U_2}}$ is $C^p$ between open subsets.\\

    $F_2(F_1(p))$ induces another pair of charts $(V_i,\psi_i)\in X_i$ for $i=2,3$. Since $F_2$ is smooth, it is continuous. $F_1^{-1}\circ F_2^{-1}(V_3)$ is open in $X_1$, and we can shrink all of our charts so that $F_2F_1(U_1)$ is contained in $V_3$. Finally, because $C^p$ morphisms between open subsets of Banach spaces is closed under composition, $F_{U_1\cap F_1^{-1}F_2^{-1}(V_3), V_3}$ is smooth.
\end{proof}
\begin{remark}[Concluding remarks]
    Manifolds hereinafter will be assumed of class $C^p$, where $p\geq 0$. If $(U,\varphi)$ is a chart in the structure of $X$, we will simply say $(U,\varphi)$ is in $X$; or $(U)$ is in $X$.
\end{remark}
\topheader{Tangent spaces}
The next question that we will address is taking derivatives of smooth maps between manifolds. There is no reason to demand $C^p$ smoothness between maps, or even a $C^p$ category of manifolds if we cannot borrow something more other than the morphisms on open sets. \textbf{In this section, all manifolds will be of class $C^p$ for $p\geq 1$.}\\

Suppose $U$ is an open subset of $E$ and $f: U\to Y$ is $C^p$ for $p\geq 1$. The derivative $Df(x)$ is a linear map $E\to F$, not from $U$ to $F$ ($U$ might not even be a vector space). This suggests the 'derivative' of a morphism $F: X\to Y$ between manifolds can in some sense be interpreted as the \emph{ordinary derivative} of its coordinate representation $DF_{U,V}(\hat{p})$, adhering to our principle of using open sets.\\

But there is a problem with this 'derivative': it gives different values for different charts. With infinitely many charts in $X$ and $Y$, this definition becomes useless. To see this, let $X$ be a manifold modelled on $E$ and $p\in X$. If $f: X\to Y$ is a morphism, and $(U_1,\varphi_1)$, $(U_2,\varphi_2)$ are charts defined about $p$ such that the representations $f_{U_1, V}$ and $f_{U_2, V}$ are morphisms. Writing $p_i = \varphi_i (p)$, $U_{1,2} = U_1\cap U_2$ and
\begin{equation}\label{eq:tangent-space-overlap-motivation}
    \varphi_{1,2}=\varphi_2\varphi_{1}^{-1}: \varphi_1(U_{1,2})\to \varphi_2(U_{1,2})
\end{equation}
(because the map in \cref{eq:tangent-space-overlap-motivation} goes from the domain $U_1$ to $U_2$), a simple computation yields \cref{eq:tangent-space-quotient-motivation}.
\begin{align}
    Df_{U_1, V}(p_1)(v) &= D(\psi f \varphi_2^{-1}\varphi_2\varphi_1^{-1})(p_1)(v) \nonumber\\
    &= Df_{U_2,V}(p_2)\biggl(D\varphi_{1,2}(p_1)(v)\biggr)\nonumber\\
    &= Df_{U_2,V}(p_2)\circ D\varphi_{1,2}(p_1)\cdot (v)\label{eq:tangent-space-quotient-motivation}
\end{align}
where $\cdot(v)$ denotes the evaluation at $v\in E$, and is assumed to be left associative over composition. The computation in \cref{eq:tangent-space-quotient-motivation} suggests that interpreting the derivative by pre-conjugation is dependent on the chart being used to interpret the derivative. In fact, $D\varphi_{1,2}(p_1)$ can be replaced with any toplinear isomorphism on $E$ (relabel $\varphi_2 = A\varphi_1$ where $A$ is any linear automorphism on $E$), so the right hand side of \cref{eq:tangent-space-quotient-motivation} can be interpreted as $Df_{U_2,V}(p_2)(w)$ where $w$ is any vector in $E$. 

\begin{definition}[Concrete tangent vector]\label{def:concrete-tangent-vector}
    Suppose $k\geq 1$, $X$ a $C^k$-manifold on $E$, and $p\in X$. If $(U,\varphi)$ is any chart containing $p$, for each $v\in E$ we call $(U,\varphi,p,v)$ a \emph{concrete tangent vector at $p$} that is \emph{interpreted} with respect to the chart $(U,\varphi)$. The disjoint union of concrete tangent vectors, as shown in \cref{eq:concrete-tangent-space}

    \begin{equation}\label{eq:concrete-tangent-space}
        T_{(U,\varphi,p)}X = \bigcup_{v\in E}\{(U,\varphi,p,v)\}\cong E
    \end{equation}

    is called the \emph{concrete tangent space at $p$} interpreted with respect to $(U,\varphi)$; and it inherits a TVS structure from $E$.
\end{definition}

Fix a point $p$ in a manifold $X$. Suppose $(U_i,\varphi_i)$ are charts containing $p$, from \cref{eq:tangent-space-quotient-motivation} there exists a natural (toplinear) isomorphism between the concrete tangent spaces, namely 
\begin{equation}\label{eq:concrete-tangent-vector-relation}
    (U_1,\varphi_1,p,v_1)\sim (U_2,\varphi_2,p,v_2)\qqtext{iff} v_2 = D\varphi_{1,2}(p_1)(v_1)
\end{equation}
where $p_i = \varphi_i (p)$. The right member of \cref{eq:concrete-tangent-vector-relation}  is the derivative of a transition map --- which is a toplinear automorphism on $E$. Hence $D\varphi_{1,2}(p_1)$ defines a toplinear isomorphism between $T_{(U_1,\varphi_1,p)}X$ and $T_{(U_1,\varphi_2,p)}X$. With this, we define the primary object of our study.
\begin{definition}[Tangent vector]\label{def:tangent-vector}
    A \emph{tangent vector} (or an \emph{abstract} tangent vector) at $p$ is defined as an equivalence class of concrete tangent vectors at $p$, under the relation in \cref{eq:concrete-tangent-vector-relation}.
\end{definition}

\begin{definition}[Tangent space]\label{def:tangent-space}
    The \emph{tangent space} at $p$, denoted by $T_p X$ is the set of all tangent vectors at $p$. It is toplinearly isomorphic to the model space $E$.
\end{definition}

\begin{definition}[Differential of a morphism]\label{def:differential-of-a-morphism}
    Let $X$ and $Y$ be modelled on the spaces $E$ and $F$. If $f$ be a morphism between $X$ and $Y$, the \emph{differential of $f$ at $p$} is the unique linear map denoted by
    \begin{equation}\label{eq:differential-of-a-morphism}
        df(p)=df_p: T_p X\to T_{f(p)} Y
    \end{equation}
    Whose action is characterized by the following:
    \begin{itemize}
        \item if $(U,\varphi)$ and $(V,\psi)$ are any pair of charts that satisfy the morphism condition in \cref{eq:smoothness} about $p$,
        \item if $v\in T_p X$ is represented by $\qty\Big(U,\varphi,p, \hat{v})$,
        \item then $df(p)(v)\in T_{f(p)}Y$ is represented by $\qty\Big(V,\psi, f(p), Df_{U,V}(\hat{p})(\hat{v}))$
    \end{itemize}
\end{definition}
\begin{note}[Interpretation using co-product]
    There is another way of interpreting the construction above. Each concrete tangent space is toplinearly isomorphic to $E$, the projection maps onto $\{p\}$ and $E$ can be glued together using the universality of the coproduct, where $\{p\}$ is interpreted as a $0$-dimensional vector space. The construction of $T_pM$ follows by invoking the property of the quotients.
\end{note}
\begin{remark}[Omission of chart in concrete representation]\label{rmk:omission-of-chart-in-concrete-rep}
    If $p$ is a point on a manifold $X$, $v\in T_p M$, we sometimes say $(U,\hat{v})$, or $\hat{v}$ is an interpretation of $v$ if it is clear $(U,\varphi)$ is a chart in $X$. If $X=E$, we will \emph{identify} $v$ with its concrete representation in the \emph{standard chart} $(E,\id{E})$. The standard representation of a tangent vector is written with a bar on top: $\cl{w}$ is the \emph{standard representation}, or \emph{standard interpretation} of $w$.\\

    Furthermore, we also write $(\hat{p},\hat{v})$ where $\hat{p} = \varphi (p)$ and $(U,\varphi, p, \hat{v})$.
\end{remark}
\begin{remark}[Morphisms between $C^k$, $C^p$ manifolds]\label{rmk:morphism-cp-ck}
    Let $X$ be a $C^k$-manifold, and $Y$ a $C^p$ manifold, where $k,p\geq 0$. A morphism between $X$ and $Y$ is a map $f: X\to Y$ such that each point $p\in X$ admits a coordinate representation 
    \begin{equation}\label{eq:morphism-cp-ck}
        f_{U,V}\in C^{\min(p,k)}(\hat{U},\hat{V})
    \end{equation}
    If $\min(p,k)\geq 1$, then we define its differential as in \cref{def:differential-of-a-morphism}.
\end{remark}
% Recap of this section, the tangent space at a point is defined to be such that when the chart in the codomain is fixed, the coordinate interpretation of the derivative is independent of the chart used. This is done by gluing together the inputs. Now, we let the chart in the codomain to vary, and we also see that it is independent of the interpretation used.
% Let $F$ be a morphism between manifolds $X$ and $Y$. Because the equivalence relation in eq3 is compatible with linear operations. ie: if we relabel (U_1,v_1) = (U_1,\varphi_1, p, v_1), similarly for U_2.
% (U_1,v_1)\sim (U_2,v_2)\qqtext{and}(U_1,w_1)\sim(U_2,w_2), then their difference and their dilations will be \sim as well.
% So, DF_{, V} is well defined, and is the unique linear map by the property of the quotients.
% on the side of the codomain, we just have to compose DF_{, V} with the map that sends each concrete tangent vector to its abstract tangent vector. This map is linear.
\topheader{Curves}
In the previous section, we motivated the definition of $T_pX$ using the computation of the derivative of a morphism from $X$. Dually, the tangent space allows us compute the derivatives of morphisms into $X$ in a coordinate independent manner.\\

Let $J_\varepsilon = (-\varepsilon, +\varepsilon)$ be an open interval in $\real$. Viewing $J_\varepsilon$ as a manifold, the morphisms $\gamma: J_\varepsilon\to X$ are \emph{curves in $X$} and $\gamma(0)$ is called the \emph{starting point of $\gamma$}.
\begin{definition}[Velocity of a curve]\label{def:velocity-of-a-curve}
    Let $\gamma$ be a curve in $X$ and $t\in J_\varepsilon$. The \emph{velocity} at $t$, denoted by $\gamma'(t)$ --- is the tangent vector with representation $d_{J_\varepsilon,V}\gamma(\cl{1})$; where $(J_\varepsilon,\cl{1})$ is a concrete tangent vector in $T_t J_\varepsilon$.
\end{definition}
\begin{wts}[Tangent vectors are velocities]\label{prop:tangent-vectors-are-velocities}
    Let $p$ be a point on a manifold $X$. For every tangent vector $v\in T_p X$, there exists a curve starting at $p$ whose velocity is $v$.
\end{wts}
\begin{proof}
    Find a chart $(U)$ in $X$ where $\hat{p}=0$. Such a chart exists, because translations and dilations are $C^p$ isomorphisms. If the tangent vector $v$ has interpretation $\hat{v}$ in $U$, there exists $\varepsilon>0$ so small that the range of $\hat{\gamma}$, as defined \cref{eq:velocity-curve-gamma-coord-rep}, lies in $\hat{U}$
    \begin{equation}\label{eq:velocity-curve-gamma-coord-rep}
        \hat{\gamma}: J_\varepsilon\to \hat{U}\quad \gamma(t) = \int_0^t \hat{v}dt
    \end{equation}
    $\hat{\gamma}$ is a curve in $\hat{U}$ starting at $\hat{p}$ with velocity $\hat{v}$. Defining $\gamma$ as the composition of $\hat{\gamma}$ with the chart inverse finishes the proof.
\end{proof}



% \fchapter{2: Submanifolds}
% Let X be a topological space, a subset $S\subseteq X$ is locally closed if every point $p\in S$ admits a neighbourhood $U\osub X$ such that $U\cap S$ is closed in $U$.

%wts: all locally closed sets are the intersection of a closed set and an open set.

% Let U_\alpha cover our locally closed S, such that U_\alpha\cap S is a closed subset of $U_\alpha$, so that $U_\alpha\setminus S$ is open in $U_\alpha$. Since $U_\alpha$ is open (Munkres Lemma 16.2) implies $U_\alpha\setminus S$ is open in $X$.

% Take the union over all such $U_\alpha\setminus S$, since $U_\alpha$ covers $S$. Denote $\bigcup U_\alpha = U$. And $U\setminus S$ is open in $X$. Its complement relative to $X$ is closed. And

% X\setminus (U\setminus S) = U^c + S is closed in $X$. 
% Take the intersection with $U$, which is an open subset of $X$,
% U\cap [X\setminus (U\setminus S)] = U\cap (U^c + S) = U\cap S = S
% Therefore $S$ is the intersection between an open set and a closed set.

% Embedded submanifolds are precisely the subsets of a manifold $X$ where we can pull existing charts (if we can 'drop' some of the coordinates) from the smooth structure of $M$. Since these charts cover $S$, it defines an atlas: and the uniqueness of the structure (which includes the topology) follows.

% Lang: Fix a manifold $X$ (of class $C^k$), and a subset $S\subseteq X$. For every $p\in S$, we can find a chart $(U,\varphi)$ in the structure of $X$ containing the point $p$, such that

% U is isomorphic, through \varphi to a product manifold $V_1\times V_2$. Where $V_i$ are open subsets of Banach space $E_i$. And that

% \varphi(U\cap S) = V_1\times {a_2}, where $a_2$ is some point in $V_2$. So $U\cap S$ belongs precisely to some fiber. The isomorphism above should be interpreted in the $C^k$ manifold sense. 

%Because open subsets of Banach spaces are submanifolds. (have we proven this yet?)

% Lee interpretation: 
% Theorem 5.16: every embedded submanifold is locally a levle set. A subset $S\subseteq X$ is an embedded submanifold if and only fi for every $p\in S$ there exists a chart in the smooth structure on $$ such tht $U\cap S$ is precisely a level set of a morphism $f_{U\cap S}$ with codomain $\real^{n-k}$. In this case we say $S$ is an embedded submanifold of dimension $k$.

% The dimension of the codomain of the local defining function of the embedded submanifold is the codimension of the embedded submanifold


\topheader{Splitting}
Recall: if $W$ is a vector space and $W_1$, $W_2$ are linear subspaces of $V$. $W_2$ is the vector space complement of $W_1$ (resp. with the indices reversed) if 
\[W_1 + W_2 = W,\qqtext{and} W_1\cap W_2 = 0\]
\begin{definition}[Splitting in $E$]\label{def:splitting-subspace}
A linear subspace $E_1$ splits in $E$ if both $E_1$ and its vector space complement $E_2$ are closed, and the addition map $\theta: E_1\times E_2\to E$ given by 
\[
    \theta(x,y)= x + y\quad\text{is a toplinear isomorphism.}
\]
We sometimes refer to the vector space complement of $W_1$ as its \emph{linear complement}.
\end{definition}
\begin{remark}[Every linear subspace splits in finite dimensions.]
Every finite dimensional or finite codimensional linear subspace of $E$ splits. If $E$ is finite dimensional, then every linear subspace splits.    
\end{remark}
%
%
% If $\lambda\in L(E,F)$ is injective, we would like to describe the situation where we can think $E$ being toplinearly isomorphic to its range, similar to the matrix canonical form $\begin{bmatrix}I_{k} & 0_{k\times (n-k)} \end{bmatrix}$. This requires the range of $\lambda$ to be a toplinear subspace, and we obtain the following.
%
\begin{definition}[Splitting in $L(E,F)$]\label{def:splitting-clm}
A continuous, injective linear map $\lambda\in L(E,F)$ \emph{splits} iff its range splits in $F$. Alternatively, $\lambda$ splits iff there exists a toplinear isomorphism $\alpha: F\to F_1\times F_2$ such that $\lambda$ composed with $\alpha$ induces a toplinear isomorphism from $E$ onto $F_1\times 0$ --- which we identify with $F_1$. 
\end{definition}
%
%

%
%
For the next few definitions, $X$ and $Y$ will be manifolds.
\begin{definition}[Immersion]\label{def:immersion}
    A morphism $f\in\Mor(X,Y)$ is an \emph{immersion} at a point $p\in X$ if there exists a coordinate representation about $f_{U,V}$ such that 
    \begin{equation}\label{eq:immersion-at-point}
        Df_{U,V}(\hat{p})\quad\text{is injective and splits.}
    \end{equation}
    The morphism $f$ is called an immersion if \cref{eq:immersion-at-point} holds at every $p$.
\end{definition}
%
%
\begin{definition}[Submersion]\label{def:submersion}
    A morphism $f\in\Mor(X,Y)$ is an \emph{submersion} at a point $p\in X$ if there exists a coordinate representation about $f_{U,V}$ such that 
    \begin{equation}\label{eq:submersion-at-point}
        Df_{U,V}(\hat{p})\quad\text{is surjective and its kernel splits.}
    \end{equation}
    The morphism $f$ is called an submersion if \cref{eq:submersion-at-point} holds at every $p$.
\end{definition}
%
%
%
\begin{remark}[Similarity to Matrix Canonical Forms]
    Suppose $X$ and $Y$ are finite dimensional and if $f$ is an immersion (resp. submersion), there exist a coordinate representation about each $p\in X$ such that \cref{eq:immersion-matrix-form} (resp. \cref{eq:submersion-matrix-form}) holds.
    \begin{align}
        D\hat{f}(\hat{p}) &= \begin{bmatrix}
            \id{m\times m} \\
            0_{n-m\times m} 
        \end{bmatrix}\label{eq:immersion-matrix-form}\\[2ex]
    D\hat{f}(\hat{p}) &= 
        \begin{bmatrix}
            \id{n\times n} & 0_{n\times m-n}
        \end{bmatrix}\label{eq:submersion-matrix-form}
    \end{align}
\end{remark}    
%
%
%
\begin{definition}[Embedding]\label{def:embedding}
    A morphism $f\in\Mor(X,Y)$ is an \emph{embedding} if it s a immersion and it is a homeomorphism onto its range.
\end{definition}

\begin{definition}[Toplinear subspace]\label{def:toplinear-subspace}
    Let $E$ be a Banach space, a \emph{toplinear subspace} (of $E$) is a closed linear subspace $E_1$ which splits in $E$.
\end{definition}
%%%

%%%
%
% End of splitting Chapter
%
\topheader{Submanifolds}
Before we state the definition of a submanifold, it is important to recapitulate the construction of a manifold $X$.
\begin{enumerate}
    \item Given a non-empty set $X$ and an atlas modelled on a space $E$.
    \item The purpose of each chart in the atlas is to borrow open subsets $\hat{U}\osub E$. If we single out a single chart, \textbf{the construction is entirely topological}. It is of little importance \emph{how} the individual chart domains $U$ are mapped onto $\hat{U}$,
    \item Each chart is in \textbf{bijection with its range}, which is an open subset of $E$, and
    \item the transition maps $\varphi_{\beta\alpha^{-1}}$ are \textbf{morphisms between open subsets of $E$}.
\end{enumerate}
In the spirit of borrowing definitions and properties from existing objects, it makes (functoral) sense a submanifold $S$ should be modelled a linear subspace of $E_1$ of $E$. The natural charts we can borrow from the structure of $X$ are those with the 'other coordinates' muted. If $(U,\varphi)$ is a chart whose domain intersects $S$, the restriction of $\varphi$ onto $U\cap S$ should be in bijection with an open subset of $E_1$. 
\begin{equation}\label{eq:motivation-submanifold}
    \varphi(S\cap U) = \hat{U}_1\times ?,\quad \hat{U}_1\osub E_1
\end{equation}
There is a problem with \cref{eq:motivation-submanifold} however, $\varphi$ is a $C^p$ isomorphism onto $\hat{U}$; not onto open subsets of the product space $E_1\times E_2$. An easy fix to this would be to require $E_1$ \textbf{to split in $E$} (and shrinking $U$ using a basis argument). Let $\alpha$ be a $C^p$ isomorphism between $E$ and $E_1\times E_2$. \Cref{eq:motivation-submanifold} becomes
\begin{equation}\label{eq:motivation-submanifold-modified}
    \alpha\varphi(S\cap U) = \hat{U}_1\times a_2\qqtext{where} \hat{U}_1\osub E_1 \text{ and }a_2\in E_2
\end{equation}
Identifying $\hat{U}$ with $\alpha(\hat{U})$, and requiring $U_1\times a_2$ to be in $\alpha(\hat{U})$, we arrive at the following definition.
\begin{definition}[Submanifold]
    Let $X$ be a manifold, and $S$ a subset of $X$. We call $S$ a \emph{submanifold} of $X$ if there exist split subspaces $E_1$, $E_2$ of $E$; such that, every $p\in S$ is contained in the domain of some chart $(U,\varphi)$ in $X$. Where
    \begin{equation}\label{eq:submanifold-slice-chart-splits}
        \varphi: U\to \hat{U}\cong \hat{U}_1\times \hat{U}_2,\qqtext{where} U_i\osub E_i\quad i=1,2
    \end{equation}
    and there exists an element $a_2\in \hat{U}_2$
    \begin{equation}\label{eq:submanifold-slice-chart-level-set}
        \varphi(U\cap S) = \hat{U}_1\times a_2
    \end{equation}
\end{definition}
We call a chart satisfying \cref{eq:submanifold-slice-chart-splits,eq:submanifold-slice-chart-level-set} a \emph{slice chart} of $S$; to simplify what follows, we write $\varphi^i = \operatorname{proj}_i\varphi$ for $i = 1,2$ for any slice chart $(U)$. Given that $\operatorname{proj}_i$ is a morphism between open subsets of Banach spaces, $\varphi^i$ is again a morphism. In particular, $\varphi^1$ is in bijection from $U^s=U\cap S$ onto $\hat{U}_1$; the latter being an open subset of $E_1$. To show $S$ is indeed a manifold it remains to show the collection of charts in \cref{eq:submanifold-induced-atlas} forms a $C^p$ atlas modelled $E_1$, which we will prove in \cref{prop:structure-of-submanifold}
\begin{equation}\label{eq:submanifold-induced-atlas}
    \acal = \bigset{(U^s, \varphi^1),\:\: (U,\varphi)\text{ is a slice chart of }S}
\end{equation}
\begin{wts}[Structure of a submanifold]\label{prop:structure-of-submanifold}
    If $S$ is a submanifold of $X$, \cref{eq:submanifold-induced-atlas} defines a $C^p$ atlas over the space $E_1$. The manifold $S$ has a topology that coincides with the subspace topology, and the inclusion map $\iota_S: S\to X$ is a morphism and a homeomorphism onto its range.
\end{wts}
\begin{proof}
    Each of the charts in \cref{eq:submanifold-induced-atlas} is in bijection with an open subset of $E_1$. Let $(U^s_{\alpha}, \varphi^1_\alpha)$ and $(U^s_{\beta}, \varphi^1_\beta)$ be overlapping charts. Writing $U^s_{\alpha\beta} = U^s_{\alpha}\cap U^s_{\beta}$ as usual, and the transition map $\varphi^1_{\beta\alpha^{-1}} = \varphi^1_{\beta}(\varphi^1_{\alpha})^{-1}$ from $\varphi_{\alpha}^1(U_{\alpha\beta}^s)$ to $\varphi_{\beta}^1(U_{\alpha\beta}^s)$. \Cref{eq:submanifold-slice-chart-level-set} tells us there exists $a_2\in \hat{U}_{2,\alpha}$ and $b_2\in \hat{U}_{2,\beta}$, that can help us recover the original chart. Identifying $a_2$ (resp. $b_2$) with the constant function $(p\mapsto a_2)$ for $p\in U_{\alpha}^s$, we get \cref{eq:submanifold-chart-recovery}.
    \begin{equation}\label{eq:submanifold-chart-recovery}
    \varphi^1_{\alpha}\times a_2 = \varphi_{\alpha}\vert_{U^s_{\alpha}}
    \end{equation}
    (resp. $\varphi^1_{\beta}\times b_2 = \varphi_{\beta}\vert_{U^s_{\beta}}$). The transition map is given by
    \begin{equation}\label{eq:submanifold-transition-map-1}
    \varphi^1_{\beta}\circ (\varphi^1_{\alpha})^{-1} = \operatorname{proj}_{1,\beta}\varphi_{\beta}\vert_{U^s_{\beta}}\qty(\varphi_{\alpha}\vert_{U^s_{\alpha}})^{-1}\qty(\operatorname{proj}_{1,\alpha}\vert_{U^s_{\alpha}})^{-1}    
    \end{equation}
    We can combine the two middle terms into $\varphi_{\beta}\varphi_{\alpha}^{-1}\vert_{U^s_{\alpha\beta}}=\varphi_{\beta\alpha^{-1}}\vert_{U^s_{\alpha\beta}}$. Which is a $C^p$ isomorphism, because the domain (resp. codomain) of $\varphi_{\alpha}(U^s_{\alpha\beta})$ (resp. $\beta$) is given by \cref{eq:submanifold-slice-chart-level-set}. Suppressing the restrictions onto $U^s_{\alpha\beta}$, we have
    \[
        \varphi_{\alpha}(U^s_{\alpha\beta}) = \qty(\hat{U}_{1,\alpha}\cap\hat{U}_{1,\beta})\times a_2 \qqtext{and}\varphi_{\beta}(U^s_{\alpha\beta}) = \qty(\hat{U}_{1,\alpha}\cap\hat{U}_{1,\beta})\times b_2
    \]
    The middle term in \cref{eq:submanifold-transition-map-1} then becomes
    \[
        \varphi_{\beta\alpha^{-1}} = (\varphi^{1}_{\beta\alpha^{-1}}, (a_2\mapsto b_2))
    \]
    Which is a $C^p$ isomorphism. The other terms in \cref{eq:submanifold-transition-map-1} are either projections or products of isomorphisms with constant functions, therefore \cref{eq:submanifold-induced-atlas} forms an atlas.\\

    Let us use $\iota_{S}: S\to X$ to represent the inclusion map and consider a fixed point $p\in S$. It is always possible to identify a slice chart $(U,\varphi)$ for the structure of $X$ that contains $p =\iota_S(p)$. By definition of the atlas on $S$, \cref{eq:submanifold-induced-atlas}, this induces a 'truncated' chart $(U^s,\varphi^1)$. \\
    
    Observing that $\iota_S(U^s) = \iota_S(U\cap S)$ lies within $(U,\varphi)$, the morphism criteria in \cref{eq:smoothness} is satisfied. Computing the coordinate representation of $\iota_S$, we obtain \cref{eq:submanifold-inclusion-map-coordinate-representation}.
    \begin{equation}\label{eq:submanifold-inclusion-map-coordinate-representation}
        (\iota_S)_{U^s,U} = \varphi\iota_S(\varphi^1)^{-1} = \id{\hat{U}_1}\times a_2
    \end{equation}
    \Cref{eq:submanifold-inclusion-map-coordinate-representation} shows that the coordinate representation of $\iota_S$ is a local isomorphism. Since the inclusion map is a bijection and continuous, and the coordinate representation of $\iota_S^{-1}$ is simply the inverse \cref{eq:submanifold-inclusion-map-coordinate-representation}; $\iota_S^{-1}$ is a morphism and therefore continuous.
\end{proof}
%
%
\begin{remark}[Pairs of slice charts]\label{rmk:pairs-of-slice-charts}
    If $p$ is a point on a submanifold $S$, we refer to a \emph{pair of slice charts} containing $p$ as the pair $(U^s,\varphi^1)$ and $(U,\varphi)$ in the structure of $S$ and $X$. 
\end{remark}
%
%
\Cref{prop:structure-of-submanifold} shows every point $p\in S$ is in the domain of a pair of slice charts. The inclusion map $\iota_S$ has coordinate representation \cref{eq:submanifold-inclusion-map-coordinate-representation}. Computing its ordinary derivative we obtain \cref{eq:derivative-of-inclusion-map-in-slice-coordinates}.
\begin{equation}\label{eq:derivative-of-inclusion-map-in-slice-coordinates}
    D(\iota_S)_{U^s,U}(\hat{p}): T_{(U^s,\varphi^1,p)}\longrightarrow T_{(U,\varphi,p)}\qqtext{and} D(\iota_S)_{U^s,U}(\hat{p}) = \id{E_1}\times 0
\end{equation}
%
which is a toplinear morphism between concrete tangent spaces and has a simple representation of 'adding zeroes' at the end of a vector $\hat{v}\in E_1$. 
%
\begin{definition}[Exterior tangent space of $S$]
    The \emph{exterior tangent space} of a point $p\in S$ is the image of $T_p S$ under $d\iota_S(p)$,
    \begin{equation}\label{eq:exterior-tangent-space}
        T_p^{ext}S = d\iota_S(p)(T_p S)
    \end{equation}
    which is a toplinear subspace of $T_p X$. 
\end{definition}

% The differential of the inclusion map $\iota_S: S\to X$ allows us to characterize $T_pS$ up to an isomorphism. 
% First, we need a few definitions.
% We now define the notion of the \emph{exterior tangent space of $S$}
% The representation of a tangent space in a chart, 

% \begin{definition}[Concrete tangent space]
%     Let $p$ be a point on a manifold $X$. The \emph{concrete tangent space} of $p$, with respect to a chart $(U,\varphi)$ is \cref{eq:concrete-tangent-space-eq}
% \end{definition}

% Functoral properties of the differential
% Tangent space functor, maps a pointed manifold to a tangent space, and 
% the action on the morphisms on pointed manifold, 'lifts' it to be a differential.

% Fiber Bundle: notebook page 41, 42
%\topheader{Properties of Differentials}
% 

\topheader{Vector Bundles}
Our goal this section is to construct a vector bundle of a manifold $X$, which is the proper setting to study vector / covector fields, and later different forms. % We want to make the idea of a map from a manifold to its tangent space rigorous, and to define what it means for such a map to be a morphism. 

% Let $X$ be a manifold modelled on a space $E$, and fix another Banach space $F$. If for every point $p\in X$, the space $W_p$ is toplinearly isomorphic to $F$, how can we glue together these Banach spaces to produce a $C^p$ manifold? One way to proceed would be to take the coproduct $W = \coprod_{p\in X}W_p$. This gives us a canonical projection $\pi: W\to X$.\\

% Recalling our construction procedure of the manifold, we want to cover the space $W$ with open subsets of Banach spaces. \\

% Restart

Let $X$ be a class $C^p$ manifold modelled on a space $E$, and $F$ another Banach space. Suppose for each $p$, the set $W_p$ is toplinearly isomorphic to $F$ at for each $p$, then we call $W_p$ an $F$-\emph{fiber} at $p$. The set-theoretic coproduct of all such $W_p$ as in \cref{eq:vector-bundle-coproduct} is called a \emph{coproduct of $F$-fibers modelled over $X$}.
\begin{equation}\label{eq:vector-bundle-coproduct}
    W = \coprod_{p\in X}W_p\qqtext{comes with} \pi: W\to X,\quad \pi^{-1}(p) = W_p
\end{equation}
 It turns out the natural way of making $W$ a manifold would be to steal open sets from \emph{both} $E$ and $F$ --- in this case, sets of the form $\hat{U}\times F$. We sometimes write $\wig{U}$ instead of $\pi^{-1}(U)$ for brevity, and $\wig{p}$ in place of $\pi^{-1}(p)$. The next few definitions will ring a few bells.
%
% LOCAL TRIVIALISATIONS
%
\begin{definition}[Local trivialisation]\label{def:local-trivialisation}
    Let $W$ be as in \cref{eq:vector-bundle-coproduct}. A \emph{local trivialisation} of $W$ is a tuple $(\wig{U}, \Phi)$, such that the diagram in \cref{fig:local-trivialisation} commutes, and
    \begin{itemize}
        \item $U\subseteq X$ is open in $X$, and for each $p\in U$,
        \item $\Phi\vert_{\wig{p}}$ is in bijection with $W_p = F$.
    \end{itemize}
\end{definition}
%
% COMPATIBILITY CRITERION
%
\begin{definition}[Compatibility between trivialisations]\label{def:compatibility-local-trivialisations}
    Let $(\wig{U},\Phi)$ and $(\wig{V},\Psi)$ be local trivialisations of $W$, they are called $C^k$-compatible if $U\cap V=\varnothing$, or both of the following hold:
    \begin{itemize}
        \item for each $p\in U\cap V$ --- the restriction of $\Psi\circ\Phi^{-1}$ onto the fiber of $p$ --- $(\Psi\circ\Phi^{-1})\vert_{\wig{p}}$ is in $\Laut(F,F)$, and
        \item the map $\theta: U\cap V\to L(F,F)$ as defined by \cref{def:vb-transition-function}, is a $C^k$ morphism into the Banach space $L(F,F)$.
        \begin{equation}\label{def:vb-transition-function}
            \theta(p) = (\Psi\circ\Phi^{-1})\vert_{\wig{p}}
        \end{equation}
        (equivalently, we can require $\theta$ be $C^k$ as a map into $\Laut(F)$).
    \end{itemize}
\end{definition}

%
% TRIVIALISATION COVERING
%
\begin{definition}[Trivialisation covering]\label{def:trivialisation-covering}
    Let $W$ be a coproduct of $F$-fibers over $X$. A \emph{$C^k$ trivialisation covering of $W$} is a collection of pairwise $C^k$-compatible local trivialisations $\{(\wig{U}_{\alpha},\Phi_{\alpha})\}$ where $\{U_\alpha\}$ is an open cover of $X$.
\end{definition}
%
% VECTOR BUNDLE
%
\begin{definition}[Vector bundle]\label{def:vector-bundle}
    Let $X$ be a $C^p$ manifold over $E$, and let $F$ be a Banach space. An \emph{$F$-vector bundle of rank $k$} is a coproduct of $F$-fibers modelled over a manifold $X$ equipped with a \textbf{maximal $C^k$ trivialisation covering}.
\end{definition}
\begin{remark}[Maximality of trivialisation covering]
    One can easily verify the compatibility condition defines an equivalence relation, thus any $C^k$- trivialisation covering \emph{determines} a maximal one.
\end{remark}
\begin{remark}[k vs. p]
    A vector bundle $W$ over $X$ can be of a different class than $X$, a morphism between $C^k$ and $C^p$ manifolds are precisely the maps whose coordinate representation about every point is $C^{\min(p,k)}$; see \cref{rmk:morphism-cp-ck} for details.
\end{remark} 
% https://q.uiver.app/#q=WzAsMyxbMCwwLCJcXHdpZ3tVfSJdLFswLDMsIlUiXSxbMywwLCJVXFx0aW1lcyBGIl0sWzAsMSwiXFxwaSIsMl0sWzAsMiwiXFxQaGkiXSxbMiwxLCJcXHByb2pfVSJdXQ==
    \begin{figure}
\centering
\begin{tikzcd}
{\wig{U}} &&& {U\times F} \\
\\
\\
U
\arrow["\pi"', from=1-1, to=4-1]
\arrow["\Phi", from=1-1, to=1-4]
\arrow["{\proj_1}", from=1-4, to=4-1]\end{tikzcd}
\caption{Local Trivialisation}
\label{fig:local-trivialisation}
\end{figure}
% We call the map $\theta$ in \cref{def:vb-transition-function} the \emph{transition function}.
The above definitions calls for some commentary, our end goal is to make an arbitrary vector bundle $W$ a $C^p$ manifold. Open sets will still be our primary 'topological' data. But in order to make $W$ as compatible to $X$ as possible, the eventual manifold structure we will put on $W$ will \textbf{embed the structure of $X$ into $W$}. This is the same argument as in the submanifold case but with the roles of $X$ and $S$ reversed.\\

Suppose we have a structure on $W$, then $X=\bigcup_{p\in X}\{p\}\times 0$ is a submanifold of the $W$ --- since $E$ splits in the product space $E\times F$. Let us motivate a couple of the requirements above.
\begin{itemize}
    \item \Cref{def:local-trivialisation}: 1) $U$ is required to be open because $W$ needs to inherit part of the topology, and we wish to 'steal' all of the charts in $E$ whose domain is a subset of $U$, 2) the second bullet point implies that \textbf{each $\Phi$ is in bijection with $\Phi(\wig{U}) = U\times F$, which is open in $E\times F$}.
    \item \Cref{def:compatibility-local-trivialisations}: 
    1) the overlap restricts to a toplinear isomorphism on each fiber because \textbf{it allows us to quotient out the effects of the trivialisation transitions}, by rehearsing the same 'coproduct and quotient' argument in \Cref{def:concrete-tangent-vector,def:tangent-vector,def:tangent-space}.
    2) the requirement that $\theta$ as defined in \cref{def-transition-map} be a $C^p$ morphism is inherited from the fact that, if $p\in (U_i,\varphi_i)$ for $i=1,2$. Then $\varphi_2\circ \varphi_1^{-1}$ is a $C^p$ isomorphism between $\varphi_1(U_{1,2})$ and $\varphi_2(U_{1,2})$; \textbf{whose derivative is a $C^{p-1}$ map into $\Laut(E)$ that encodes the transformation between the concrete tangent spaces}. In the notation of \cref{eq:tangent-space-overlap-motivation}, this means
    \[
        x\mapsto D\varphi_{1,2}(x)\quad\text{is in } C^{p-1}(\hat{U}_{1,2}, \Laut(E))
    \]
    In fact, the tangent bundle is a $C^{p-1}$ vector bundle (modelled on $E$) over $X$.
\end{itemize}
Suppose $W$ is an $F$-vector bundle over $X$ with the trivialisation covering $\{(\wig{U}^\alpha, \Phi_\alpha)\}$. For each $\alpha$, we can cover $U^\alpha$ using chart domains $(U^\alpha_\beta, \varphi^\alpha_\beta)$ in $X$ --- without loss of generality, we can assume $U^\alpha_\beta\subseteq U^\alpha$ by restricting the chart domain and relabelling. \\

Similar to the construction of the induced atlas of a submanifold, given a 'piece' of the original manifold $X$ --- \textbf{instead of dropping the coordinates that correspond to $E_2$, we add an $F$-component to construct a bijection with an open subset of $E\times F$}. This is shown in \cref{eq:vector-bundle-product-chart} 
\begin{equation}\label{eq:vector-bundle-product-chart}
    \wig{\varphi}^{\alpha}_\beta:\wig{U}^\alpha_\beta\longrightarrow\hat{U}^\alpha_\beta \times F \qqtext{defined by} \wig{\varphi}^{\alpha}_\beta = \qty(\varphi^\alpha_\beta\times \id{F})\circ \Phi_\alpha
\end{equation}
\begin{remark}[Hats and wiggles]
    Here, $\wig{U}^\alpha_\beta$ should be interpreted as the inverse image of the open set $U^\alpha_\beta$ through $\pi$. Similarly, $\hat{U}^\alpha_\beta$ is the image of $U^\alpha_\beta$ through $\varphi^\alpha_\beta$.
\end{remark}
The collection of charts in \cref{eq:vector-bundle-charts-ALPHA} cover $W$ with their chart domains, and each chart is in bijection with an open subset of $E\times F$.
\begin{equation}\label{eq:vector-bundle-charts-ALPHA}
    \mathcal{A} =\bigset{\qty\big(\wig{U}^\alpha_\beta,\: \wig{\varphi}^\alpha_\beta),\:  (\wig{U}^\alpha,\Phi_\alpha) \text{ is in the trivialisation covering of  } W.}
\end{equation}


\begin{wts}[Structure of a Vector Bundle]\label{prop:structure-of-vector-bundle}
    Let $X$ be a $C^p$ manifold modelled over $E$. If $W$ is a $C^k$ vector bundle modelled on $F$ over the manifold $X$, then $W$ is a $C^k$ manifold modelled on the product space $E\times F$. Furthermore:
    \begin{enumerate}
        \item The \emph{canonical projection} $\pi: W\to X$ is a morphism and a submersion.
        \item 
    \end{enumerate}
\end{wts}
% This ties into the computing the exterior derivative of a differential form.
% What is the reason for 'taking the covector field of the component function?' and replacing it with a wedge product?
% 

\fchapter{2: Coordinates}

\topheader{Introduction}
In the previous chapters, a chart $(U,\varphi)$ was often equated with its domain. We will now express a concrete tangent vector as $(\hat{p}, \hat{v})$, omitting any reference to the chart or its domain. \\

Let $X$ be a manifold and $F$ a Banach space. Consider a morphism $f \in \Mor(X,F)$ and fix a point $p \in X$, and write $q = f(p)$. By adopting the canonical interpretation $\cl{w}$ for a tangent vector $w \in T_q F$ (as discussed in \cref{rmk:omission-of-chart-in-concrete-rep}), we 

\begin{itemize}
    \item reinterpret the differential at $p$ $df_p$ as a linear map from $T_p X$ to $F$,
    \item always use the standard chart $(\id{F}, F)$ so that $\hat{f} = f_{U,F}$.
\end{itemize}

% Add some motivation for why we onlyl consider finite dimensional manifolds here.

In this context, morphisms into $\real$ almost serve as test functions in the framework of distribution theory. This requires a definition.

\begin{definition}[Function on $X$]\label{def:function-on-X}
    Let $X$ be a manifold of class $C^p$ over $\realn$ for $n,p\geq 1$. A \emph{function} on $X$ is a morphism $f: X\to\real$, where $\real$ should be interpreted as a manifold. We denote the commutative ring of functions on $X$ by $C^p(X,\real)$ or $C^p(X)$. If $U$ is an open subset of $X$, its functions are denoted by $C^p(U,\real)$ or $C^p(U)$.
\end{definition}
\textbf{For the rest of this chapter, assume all manifolds to be $C^p$-manifolds over $\realn$, where $n,p \geq 1$}. 

\topheader{Derivations}

% Some more motivation for this? 'If $f$ is only defined on an open subset of a Banach space then its derivative allows us access to the underlying Banach space structure even if $f$ is defined on a very small patch in $E$?

Let $E$ and $F$ be Banach spaces and $U\osub E$, suppose $f$ is a morphism from $U$ to $F$. If $p$ is a point in $U$, $Df(p)$ is of course a linear map from $E$ to $F$; this suggests a natural pairing $\hat{\dzz}$ of $f$ with and $(p,v)\in U\times E$ as shown in \cref{eq:derivation-natural-pairing-motivation}.
\begin{equation}\label{eq:derivation-natural-pairing-motivation}
    \hat{\dzz}: (U\times E)\times C^p(U,F)\longrightarrow F:\quad \qty\Big((p,v), f)\mapsto Df(p)(v)\in F
\end{equation}

Suppose $F = \real$ and denote pointwise multiplication on $\real$ by $m$. The above pairing trivially satisfies the product rule displayed in \cref{eq:derivations-product-rule}.
\begin{equation}\label{eq:derivations-product-rule}
    Dm(\UL{f}[k])(p)(v) = \sum_{i=\underline{k}}m(\UL{f}[i-1](p),D f_i(p)(v),\UL{f}[i+][k-i](p))
\end{equation}
where $\UL{f}[k]\in C^p(U,\real)$. Next, if $f$ is a function (from a manifold $X$) defined on an open neighbourhood $U$ of $p$. If $v\in T_p X$, the commentary in the introduction suggests a 'duality pairing' between $f$ and $(p,v)$ in the form of \cref{eq:tangent-vector-abstract-action}.
\begin{equation}\label{eq:tangent-vector-abstract-action}
    \dzz: (U\times E)\times C^p(U,F)\longrightarrow F:\quad \dzz\qty\Big((p,v),f) = df_p(v)
\end{equation}
\textbf{By definition of the differential $df_p$}, the right hand side of \cref{eq:tangent-vector-abstract-action} is representation independent, hence
\begin{equation}\label{eq:tangent-vector-concrete-action}
    \dzz((p,v), f) =  D\hat{f}(\hat{p})(\hat{v}),\quad\text{where the right member is an ordinary derivative}
\end{equation}
for any representation $(\hat{p},\hat{v})$, $\hat{f}$. We also see that $\dzz((p,v), f) = \hat{\dzz}((\hat{p},\hat{v}),\hat{f})$, which shows functions defined on $U$ are dual to $T_p X$ for each $p\in U$. We will make this notion precise when we introduce covectors.

\begin{definition}[Derivation at $p$]\label{def:derivation-at-p}
A \emph{derivation at $p$} is a \textbf{linear functional} $v$ on $C^p(U,\real)$, where $U$ is any neighbourhood of $p$; such that for $\UL{f}[k]\in C^p(U)$, \cref{eq:derivations-product-rule-abstract} holds. 
\begin{equation}\label{eq:derivations-product-rule-abstract}
    v\qty(m(\UL{f}[k])) = \sum_{i=\underline{k}}m(\UL{f}[i-1](x), v(f_i),\UL{f}[i+][k-i](x))
\end{equation}
We will denote the space of derivations at $p$ by $\dzz_p(X)$, and if $v\in \dzz_p(X)$, we say $v$ \emph{derives} $f$ for any function $f$ defined about $p$.
\end{definition}

%
%
We have shown every tangent vector is a derivation, since the product rule descends from \cref{eq:derivations-product-rule} and its computation in coordinates in \cref{eq:tangent-vector-concrete-action}.  If $X$ is finite-dimensional, \cref{prop:tangent-space-isomorphic-to-derivations-finite-dimensional} shows derivations at a point $p\in X$ are uniquely represented by a tangent vector.
%
%
\begin{wts}[$T_p X$ is isomorphic to $\dzz_p(X)$]\label{prop:tangent-space-isomorphic-to-derivations-finite-dimensional}
Let $p$ be a point on a manifold $X$, then its tangent space is isomorphic to the vector space of derivations. If $(\hat{p},\hat{v})$ is a concrete tangent vector, its derivation of $f$ computed using \cref{eq:tangent-vector-concrete-action}.
\end{wts}
\begin{proof}
    Postponed.
\end{proof}

\topheader{Boundary}

% Define the half space.
% Manifolds with boundary defined a the half space.
% Inward, outward, and tangent vectors.
    % The definition should be using the curves with domains (-\varepsilon, 0] and [0,+\varepsilon)
    %
% Recap the submanifold tangent space criterion? An abstract tangent vector is in the exterior tangent space of the submanifold if its concrete repreresentation in a slice chart is 
\end{document}