\documentclass[../main-v2-manifolds.tex]{subfiles}


\begin{document}
\fchapter{22: Symplectic Manifold}
\topheader{Symplectic Tensors}
    \begin{definition}[Billinear forms]
        Let $V$ be a vector space, a \emph{billinear form} $\omega: V\times V\to \real$ is a $2$-tensor on $V$.
    \end{definition}
    
    \begin{definition}[Characterization of billinear forms]
        Let $\omega$ be a billinear form on $V$, it is 
        \begin{itemize}
            \item \emph{symmetric} if 
            \[
                \omega(x,y) = \omega(y,x)
            \]
            \item \emph{skew-symmetric} or \emph{anti-symmetric} if 
            \[
                \omega(x,y) = (-1)\omega(y,x)
            \]
            \item \emph{alternating} if
            \[
                \omega(x,x)=0
            \]
        \end{itemize}
        If $V$ is a vector space over the field $F$ and $\operatorname{char}(F)\neq 2$, then the last two conditions are equivalent. Moreover, 
        \begin{itemize}
            \item $V$ is called an \emph{orthogonal geometry} if $\omega$ is symmetric.
            \item $V$ is called a \emph{symplectic geometry} if $\omega$ is alternating.
        \end{itemize}
    \end{definition}

    \begin{definition}[Metric vector space]
        A vector space (not necessarily finite dimensional) is called a \emph{metric vector space} if it is a orthogonal or symplectic geometry.
    \end{definition}


    \topheader{Matrices and billinear forms}

    \begin{definition}[Matrix of billinear form]
        If $B=(b_1,\ldots,b_n)$ is an ordered basis for $V$, we define the \emph{matrix representation of $\omega$} by
        \[
            \mcal(\omega) = (a_{ij}) = (\omega(b_i, b_j))
        \]
    \end{definition}
    \begin{wts}[Matrix induces a billinear form]
        Let $A = (a_{ij})$ be a matrix on $V$ with respect to some basis $B = (\Ul{b}[n])$ it is clear that $A$ induces a billinear form, on $V$ through $A(x,y) = [x]_B^TA[y]_B$, where $[\cdot]_B$ denotes the canonical isomorphism $V\cong \realn$ with respect to the basis $B$.

        \[
            [x]_B^T A [y]_B = \begin{bmatrix}
                x^1 &\ldots & x^n
            \end{bmatrix}A \begin{bmatrix}
                y^1\\
                \vdots\\
                y^n
            \end{bmatrix}
        \]
        for $x = x^ib_i$ and $y = y^jb_j$. 
    \end{wts}
    Moreover,
        \[
            A[x]_B = \begin{bmatrix}
                A(b_1,x)\\
                \vdots\\
                A(b_n,x)
            \end{bmatrix}\quad \parbox{4cm}{is a \emph{column} vector whose entries are given by applying $x$ on the second coordinate}
        \]
        and 
        \[
            [x]_B^T A = \begin{bmatrix}
                A(x,b_1) & \cdots & A(x, b_n)
            \end{bmatrix}\quad\parbox{4cm}{is a \emph{row} vector whose entries are given by applying $x$ on the first coordinate}
        \]
    Let $A_B$ be the matrix representation of $\omega$ with respect to the $B$, if $C$ is another basis on $V$, then how do we compute $A_C$? The answer is simple, recall for any vector $x\in V$, $x = x^i_Bb_i$ and $x = x^j_C c_j$, then
    \[
        [x]_B = M_{C,B}[x]_C\quad\text{for some matrix of an automorphism }M_{C,B}
    \]
    $\omega(x,y) = [x]_B^TA_B[y]_B = ([x]_C^TM_{C,B}^T)A_B(M_{C,B}[y]_C) = [x]_C^T A_C [y]_C$, then
    \begin{equation}\label{lee-chp22:congruent-matrices}
        M^T_{C,B}A_BM_{C,B} = A_C
    \end{equation}
    We can describe this relation between the two matrices $A_B$ and $A_C$ by the following
    \begin{definition}[Congruent matrices]
        Two matrices $M$ and $N$ are said to be \emph{congruent}, if there exists an invertible matrix $P$ for which
        \[
            P^TMP = N
        \]
        Congruence is an equivalence relation on the space of matrices, and the equivalence classes over congruence are called \emph{congruence classes}.
    \end{definition}
    
    \begin{wts}[Characterization of matrices using congruence]
        Let $A_1$ and $A_2$ be matrix representations of two billinear forms with respect to the basis $B$.
        
        \[
            A_1 = (A_1(b_i,b_j))_{ij}\quad A_2 = (A_2(b_i,b_j))_{ij}
        \]
        They induce the same billinear form if and only if they are congruent. 
    \end{wts}

    \begin{definition}[Alternate matrices]
        Let $M$ be a matrix with $F$-coefficients, it is \emph{alternate} if it is skew symmetric and is \emph{hollow}; meaning it has $0$s on the main diagonal. If $F=\real$ or $\operatorname{char}(F)\neq 2$, then alternate matrices are and are precisely the skew-symmetric matrices.
    \end{definition}

    \topheader{Orthogonality}
    For this section, $(V,\omega)$ will denote a metric vector space, not necessarily finite-dimensional unless we are using matrix representations.
    
    \begin{definition}[Orthogonal complements]
        A vector $x\in V$ is orthogonal to another vector $y\in V$, written $x\perp y$, if $\omega(x,y)=0$. \\

        If $V$ is an orthogonal or symplectic geometry then $\perp$ is a symmetric relation. If $E$ is a subset of $V$, we denote the \emph{orthogonal complement of $E$} by 
        \[
            E^\perp \defined \bigset{v\in V,\: v\perp E}
        \]
    \end{definition}

    \begin{definition}[Characterization of metric vector spaces]
        \begin{itemize}
            \item A nonzero vector $x\in V$ is \emph{isotropic}, or \emph{null} if $\omega(x,x)=0$
            \item $V$ is \emph{isotropic} if it contains at least one isotropic vector.
            \item $V$ is \emph{anisotropic} or \emph{nonisotropic} if for every $x\in V$, $\omega(x,x)=0\implies x=0$,
            \item $V$ is \emph{totally isotropic} (that is, symplectic if $\operatorname{char}(F)\neq 2$) if $\omega(x,x)=0$ for every vector $x\in V$. 
        \end{itemize}
        
            The first bullet point above is about vectors in $V$, while the others are properties of $V$.
        \begin{itemize}
            \item A vector $x\in V$ is called \emph{degenerate} if $x\perp V$, that is, 
            \[
                \forall y\in V,\: \omega(x,y)=0
            \]
            \item The \emph{radical} of $V$, denoted by $\rad{V}$ is the set of all degenerate vectors in $V$,
            \[
                \rad{V}\defined V^\perp  
            \]
            \item $V$ is \emph{singular} or \emph{degenerate} if $\rad{V}\neq \{0\}$,
            \item $V$ is \emph{non-singular} or \emph{non-degenerate} if $\rad{V} = \{0\}$,
            \item $V$ is \emph{totally singular}, if $\rad{V} = V$. 
        \end{itemize}
        To summarize,
        \begin{itemize}
            \item $V$ is isotropic if there exists a non-zero isotropic vector, meaning $\omega(x,x)=0$, for some $x\neq 0$,
            \item $V$ is degenerate if there exists a degenerate vector, $x\perp V$.
        \end{itemize}
    \end{definition}
    \begin{wts}[Matrix invariants under congruence]\label{matrix-invariants-under-congruence}
        Non-singularity, symmetry, and skew-symmetry are invariants under congruence.
    \end{wts}
    \begin{proof}
        
    \end{proof}
    \begin{wts}[Characterization of non-degeneracy]
        $V$ is non-degenerate if and only if every matrix representation $A$ of $\omega$ is non-singular. 
    \end{wts}
    \begin{proof}
        Suppose $V$ is non-degenerate, then let $B = (\Ul{b}[n])$ be a basis for $V$, if $A$ is the matrix representation of $\omega$ with respect to $B$, let $x$ be a non-zero vector in $V$, so $x\notin \rad{V}$
        \[
            b_i^T A [x]_B = \omega(b_i, x)\neq 0\implies A[x]_B\neq 0
        \]
        so $A$ is non-singular. If $A'$ is another matrix representation with respect to another basis $C$, by \Cref{lee-chp22:congruent-matrices} $A'$ is non-singular as well.\\

        Conversely, if every matrix representation of $\omega$ is non-singular, let $x$ be a non-zero vector in $V$, then $A[x]_B\neq 0$ is a non-zero vector so there exists some basis component  $(A[x]_B)^j$ that is non zero, and
        \[
            [b_j]_B^TA[x]_B = \omega(b_j,x)\neq 0
        \]
        therefore $V$ is non-degenerate.
    \end{proof}
    \begin{wts}[Characterisation of billinear forms from matrix representations]
        Let $\omega$ be a billinear form on $V$, if $\mcal(\omega)$ the induced matrix representation relative to any basis. Assume $V$ is a vector space over $\real$, then
        \begin{itemize}
            \item it is symmetric iff $\mcal(\omega)$ symmetric as a matrix,
            \item it is skew-symmetric, iff alternating iff $\mcal(\omega)$ is skew-symmetric as a matrix.
        \end{itemize}
    \end{wts}
    
    \begin{corollary}[Characterisation of non-singular symplectic form]\label{non-singular-symplectic-form-matrix}
        Let $(V,\omega)$ be a finite dimensional vector space over $\real$, equipped with a billinear form $\omega$. $(V,\omega)$ is a non-singular symplectic vector space iff the matrix representation of $\omega$ with respect to every basis is non-singular and skew-symmetric.
    \end{corollary}
    
    % Matrix Rep for non-singular Alternate Matrix <=> symplectic form
    
    
\topheader{Riesz Representation Theorems}
\begin{wts}
    Let $(V,\omega)$ be a nonsingular metric vector space, the map $x\mapsto x\lrcorner \omega\in V^*$ defined by
    \[
        x\lrcorner\omega = \omega(x,\cdot),\qqtext{and}(x\lrcorner\omega)(y) = \omega(x,y),\quad\forall y\in V
    \]
    is a linear isomorphism from $V$ to $V^*$. 
\end{wts}

\topheader{Isometries}
\begin{definition}[Isometry between MVS]
    Let $(V,\omega)$ and $(W, \eta)$ be metric vector spaces. An \emph{isometry} $\tau\in L(V,W)$ is a linear isomorphism that preserves the billinear form.
    \[
        \omega(u,v) = \eta(\tau u, \tau v)
    \]
\end{definition}
\begin{definition}[Orthogonal, symplectic groups]
    Let $V$ be a nonsingular metric vector space. If $V$ is an orthogonal (resp. symplectic) geometry, the set of all isometries on $V$ is called the \emph{orthogonal (resp. symplectic) group on $V$}. It is a group under composition, and is denoted by $\mathcal{O}(V)$ (resp. $\operatorname{Sp}(V)$).
\end{definition}

\topheader{Hyperbolic spaces, nonsingular completions}


\topheader{Canonical Forms}

\topheader{Symplectic Manifolds}

\topheader{Darboux's Theorem}
\begin{wts}[Lie Derivatives of Tensor Fields (along time-varying vector fields)]\label{lee-chp22:prop-22.14}
    Let $M$ be a smooth manifold. Suppose $V: J\times M\to TM$ is a smooth time-varying vector field on $M$. Denote the time-varying flow of $V$ by $\psi:\Epsilon\to M$. Let $A\in \Tau^k(M)$ be a smooth time-invariant covariant $k$-tensor field on $M$. For every $(t_1, t_0, p)\in \Epsilon$,
    \begin{equation}\label{prop-22.14-eq-1}
        \eval{\dv{t}}_{t=t_1}(\psi^*_{t,t_0}A)_p = (\psi^*_{t_1,t_0}(\mathcal{L}V_{t_1} A))_p
    \end{equation}
\end{wts}

\fchapter{Hofer book}
\begin{definition}[Symplectic vector space]
    Let $V$ be a finite dimensional vector space over $\real$. It is a \emph{symplectic vector space} if it admits a non-singular, antisymmetric billinear form $\omega: V\times V\to \real$.
    \[
        \omega(u,v) = -\omega(v,u)
    \]
    for $u,v\in V$. By the previous section on Riesz Representation, the linear map 
    \[
        \hat{\omega}:V\to V^*,\quad v\mapsto \omega(v,\cdot)
    \]
    is a linear isomorphism of $V$ onto its dual vector $V^*$. \\

    We define the \emph{standard symplectic vector space} $(\real^{2n},\omega_0)$, where $n\in \nat^+$, where 
    \[
        \omega_0(u,v) = <Ju, v>\quad J \defined\begin{bmatrix}
            0    & I_n\\ 
            -I_n & 0
        \end{bmatrix}
    \]
    where $<\cdot,\cdot>$ denotes the standard inner product on $\real^{2n}$. 
    \begin{equation}\label{std-symplectic-form}
        \omega_0(u,v) = <Ju,v> = <u, J^Tv> = u^{T}J^T v
    \end{equation}
    $J^T=-J$ by \Cref{non-singular-symplectic-form-matrix}.
\end{definition}
We will mainly deal with non-singular symplectic forms because of Riesz isomorphism.
\begin{definition}[Symplectic linear map]
    Let $(V,\omega)$ be a symplectic vector space. A linear map $F\in \Hom[V]$ is \emph{symplectic} if it preserves symplectic form $\omega$. For every $u\in V$, 
    \[
        <u, v> = <Au, Av> \defined A^*\omega(u,v)
    \]
    where $A^*: \Tau^*(V)\to \Tau^*(V)$ denotes the tensor pullback by precomposing any tensor $S$ by $A$
    \[
        \forall S\in\Tau^k(V),\quad  A^*S(\Ul{v}[k]) \defined S(\Ul{Av}[k])
    \]
    The set of linear symplectic maps on a $2n$-dimensional vector space form a group under composition. It is a Lie Group denoted by $\operatorname{Sp}(n)$.
\end{definition}
\begin{wts}[Symplectic Maps are Area-preserving]
    Let $(\real^{2n},\omega_0)$ denote the standard symplectic space. If $\varphi\in \operatorname{Sp}(n)$, then $\det \varphi = 1$. 
\end{wts}
\begin{proof}
    See page 4.    
\end{proof}

\begin{align}
(\Lambda_{s-\vert\alpha\vert}\partial^\alpha f)^{\hat{\:}} &= (1+\vert\zeta\vert^2)^{s/2 - \vert\alpha\vert/2}\cdot(\partial^\alpha f)^{\hat{\:}}\\
&= (1+\vert\zeta\vert^2)^{s/2 - \vert\alpha\vert/2}\cdot(2\pi i \zeta)^{\alpha}
\cdot \hat{f}\\
&= (2\pi i)^{\vert\alpha\vert}(1+\vert\zeta\vert^2)^{(s-\vert\alpha\vert)/2}\cdot\vert\zeta\vert^{\vert\alpha\vert}\cdot\hat{f}\\
&\leq{\vert\alpha\vert}(1+\vert\zeta\vert^2)^{s/2}\hat{f}
\end{align}


\end{document}