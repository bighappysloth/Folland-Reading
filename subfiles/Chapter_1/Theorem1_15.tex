\documentclass[../../main.tex]{subfiles}

\begin{document}
\problem{1.15}
\begin{wts}

\end{wts}
\begin{proof}

If $\{E_j\}_{j\geq 1}\subseteq \acal$ such that each $E_j = FDU(I_{ji})$ over finitely many $i$, and suppose $E_j$ are disjoint, and that $DU(E_j)\in\acal$. So that $DU(E_j) = FDU(I_{\alpha})$ for some finite collection of half-intervals $\{I_\alpha\}$. \\

We will first prove the simpler case. Suppose we have already proven:
\begin{multline}
\{E_j\}_{j\geq 1}\subseteq\acal,\: DU(E_j) = I_\alpha \in \acal\implies \\
\munot\biggl(DU(E_j)\biggr) = \sum \munot(E_j) = \munot(I_\alpha)
\end{multline}

but each $E_j$ is a FDU of $I_{ji}$, and for every $j\geq 1$, $E_j\cap I_\alpha\in\acal$ (closure under intersections, because the family of FDU of h-intervals is an algebra).\\

Thus we have a disjoint sequence whose union is one h-interval. In symbols:
\[
DU(E_j) = FDU(I_{\alpha}) \implies I_{\alpha} = DU( E_j \cap I_{\alpha})
\]

$\forall j\geq 1,\: E_j\cap I_\alpha\in\acal\implies$
\begin{align*}
\munot (FDU(I_{\alpha})) &= \sum_{\alpha<+\infty}\munot(I_\alpha)\\
&= \sum_{\alpha<+\infty} \sum_{j\geq 1} \munot(E_j\cap I_{\alpha})\\
&= \sum_{j\geq 1} \sum_{\alpha<+\infty}  \munot(E_j\cap I_{\alpha})\\
&= \sum_{j\geq 1}\munot(E_j)
\end{align*}
It is permissible to swap the two summations because we are using the supremum definition for a sum of non-negative terms. And we applied finite-additivity (see earlier), to conclude that $\sum_{j\geq 1}\sum_{\alpha}\munot(E_j\cap I_{\alpha}) = \sum_{j\geq 1}\munot(E_j)$.



\end{proof}
\newpage

\providecommand{\hil}{\mathcal{H}}

Define
\begin{itemize}
\item $\hil_1 = \bigset{(a,\: b],\: -\infty\leq a<b<+\infty}$,
\item $\hil_2 = \bigset{(a,\: +\infty),\: a\in\real\cup\{-\infty\}}$,
\item $\hil = \hil_1 + \hil_2 + \{\varnothing\}$. Where $+$ denotes the disjoint union.
\item $DU$: disjoint union, $FDU$: finite disjoint union.
\end{itemize}

Steps:
\begin{enumerate}
\item Show that $\hil$ is an elementary family.
\item Show that if $I_\alpha\in\hil_1$, then for every $I_\beta\in\hil_1\cup\hil_2$, $I_\alpha\cap I_\beta \in\hil_1$. We write this as 
\[
I_\alpha\cap\hil_1 = \hil_1,\: I_\alpha\cap\hil_2 = \hil_1
\]
\item Show that if $I_\alpha\in\hil_2$, then
\[
I_\alpha\cap\hil_1 = \hil_1,\: I_\alpha\cap\hil_2 = \hil_2
\]
\item Show that $\munot((a,\: b]) = \cl{F}(b) - \cl{F}(a)$ is well defined. (modify the proof in Folland to check for $a = -\infty$ with
\[
\cl{F}:\cl{\real}\to\cl{\real},\quad \begin{cases}\cl{F}|_{\real} &= F\\ \cl{F}(+\infty) &= \sup_x F(x),\\  \cl{F}(-\infty) &=\inf_x F(x)\end{cases}
\]
\item Show that $\munot((a,\: b]) = \cl{F}(b) - \cl{F}(a)$ is well defined for $b < +\infty$. If $E = (a,\: b]\in\acal$, then $E$ is an FDU of $\hil_1$, and $\hil_2$. So we write
\[
    E = FDU(\hil_1) + FDU(\hil_2) = FDU(\hil_1)
\]
since $E$ is bounded above, the $\hil_2$ part of the FDU must be null. Now fix $E = {FDU}_{\hil_1}(I_j) = {FDU}_{\hil_1}(I_2)$. And follow the proof in Folland to see the 'well-definedness' of $\munot$ if $E\in\hil_1$.
\item Next, suppose $E \in\hil_2$ and 
\[
    E = FDU(\hil_1) + FDU(\hil_2)
\]
Clearly $FDU(\hil_2)\neq\varnothing$, since $E$ is unbounded above, and $FDU(\hil_2)$ consists of exactly one element, so we write
\[
    E = FDU(\hil_1) + (z,\: +\infty)
\]
\item Show that $\munot((a,\: b]) = \cl{F}(b)- \cl{F}(a)$ is well defined. Hint: use the fact that if $E\in\mathcal{A}$, such that $E = FDU(E,\hil_1) + FDU(E,\hil_2)$, then $FDU(E,\hil_2)$ contains at most one element (after throwing away empty sets), then use this to deduce $E\cap I_\alpha$ has a $FDU(E\cap I_\alpha, \hil_2)$ of exactly one $\hil_2$ interval, where $I_\alpha$ participates in $FDU(E,\hil_2)$, if $E$ is unbounded above. Then take $E\setminus I_\alpha = FDU(E\setminus I_\alpha,\hil_1) = FDU(E,\hil_1)$.
\item Now show that $\munot$ is well-defined on all $E\in\acal$.
\item Continue the proof for Folland until you reach the unbounded intervals, then modify the 'right continuity argument' to add an extra $\hil_2$ interval. Let $I = \hil_1 + \hil_2 = I_\alpha + I_\beta$, meaning $I$ can be represented by at most one $\hil_1$ and $\hil_2$ interval. If $(I_k)\subseteq\hil_1\cup\hil_2$, then $\{I_k\cap I_\alpha\}\subseteq\hil_1$, and continue the proof as usual. 
\end{enumerate}
\end{document}