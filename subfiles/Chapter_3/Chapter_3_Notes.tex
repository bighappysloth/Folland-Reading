\documentclass[../../main.tex]{subfiles}

% Chapter 3 Notes
\begin{document}

\section*{Notes on Chapter 3}
\begin{wts}
    Prove two things,
    \begin{enumerate}
        \item $\limsup_{r\to R}\phi(r)=\lim_{\varepsilon\to 0}\sup_{0<|r-R|<\varepsilon}\phi(r)=\inf_{\varepsilon>0}\sup_{0<|r-R|<\varepsilon}\phi(r)$,
        \item $\lim_{r\to R}\phi(r)=c\iff \limsup_{r\to R}|\phi(r)-c|=0$
    \end{enumerate}
\end{wts}
\begin{proof}
    
\end{proof}



\newpage
\begin{wts}
    If $U\subseteq B(1,0)=\{|x|<1\}$, and $U\in\borel$, and if $m(U)>0$, then the family of sets 
    \[
    E_r = \bigset{x+ry,\:y\in U}
    \]
    shrinks nicely to $x\in\realn$.
\end{wts}
\begin{proof}
    Let $r>0$ be fixed then $\forall z\in E_r\induces z = x+ry$. Hence,
    
    \begin{align*}
        d(x,z) &= d(x, x + ry)\\
                &= |r|d(0,y)<|r|
    \end{align*}
    by translation invariance.
\end{proof}
\newpage

\begin{definition}[Signed measure]
    Let $\mcal$ be a $\sigma$-algebra and $\nu:\mcal\to [-\infty,\: +\infty]$ be a set function on $\mcal$. It is a \emph{signed measure} on $\mcal$ if 
    \begin{itemize}
        \item $\nu(\varnothing)=0$,
        \item $\nu$ assumes at most one of the values $\pm\infty$,
        \item If $\{E_j\}_{j\geq 1}$ is a countable, disjoint sequence of sets, the expression
        \[
            \sum_{j\geq 1}\nu(E_j)\quad\text{is unambiguous, and is equal to }\nu\qty(\bigcup E_j)
        \]
        More precisely, 
        \begin{itemize}
            \item if $\qty|\nu\qty(\bigcup E_j)|<+\infty$, the series $\sum \nu(E_j)$ converges absolutely,
            \item if $\nu\qty(\bigcup E_j)= \pm\infty$, the series $\sum \nu(E_j)$ diverges to $\pm\infty$ on every permutation.
        \end{itemize}
    \end{itemize}
\end{definition}

\begin{definition}[Positive, negative, null sets]
    Let $\nu$ be a signed measure on $\mcal$. A measurable set $E\in\mcal$ is called \emph{positive} (resp. \emph{negative}, \emph{null}) if every measurable subset $F\subseteq E$ satisfies $\nu(F)\geq 0$ (resp. $\nu(F)\leq 0$, $\nu(F)$=0).
\end{definition}


\begin{definition}[Mutual singularity]
    Two signed measures, $\nu$ and $\mu$ on a common $\sigma$-algebra $\mcal$ are \emph{mutually singular}, denoted by $\nu\perp\mu$ if there exists disjoint, measurable sets $E,F$ whose union is $\xx$.
    \[
        \mu \text{ is null on }E,\quad\text{ and }\nu \text{ is null on }F
    \]
\end{definition}


\end{document}