\documentclass[../../main.tex]{subfiles}

\begin{document}
\providecommand{\fn}{\{F_n\}}
\begin{wts}
    Let $\nu$ be a signed measure on $(\xx,\mcal)$. 
    \begin{itemize}
        \item If $\{E_j\}$ is an increasing sequence, $\lim_{n\to\infty}\nu(E_n) = \nu(\bigcup E_n)$, and
        \item if $\{E_j\}$ is a decreasing sequence --- provided that $\nu(E_1)\in\real$ --- then $\lim_{n\to\infty}\nu(E_n) = \nu(\bigcap E_n)$.
    \end{itemize}
\end{wts}
\begin{proof}
    Let $\nu$ be a signed measure, and $E_j\nearrow E=\bigcup E_{j\geq 1}$. This induces a disjoint sequence in $\{F_n\}$, where $F_1 = E_1$, and if $n\geq 2$,
    \[F_n = E_n\setminus\bigcup E_{j\leq n-1}\]
    The first claim follows from the definition of $\nu$, where $\sum_{n=1}^{\infty}\nu(E_n) = \lim_{n\to\infty}\sum_{j\leq n}\nu(E_j)$.\\

    Next, for any measurable $A$, $B$, where $A\subseteq B$, if $\nu(A)=\pm\infty$, then $\nu(B)=\pm\infty$. It follows that $\mu(\cap E_n)\in\real$ as well. We can produce an increasing sequence through $G_n = E_1\setminus E_n$ for $n\in\nat^+$. Then,
    \[
        \bigcup G_n = \bigcup E_1\setminus E_n = E_1\cap\qty[\bigcup E_n^c]=\qty[\bigcap E_j]^c.
    \]
    We then write $E_1 = \qty[\bigcup G_n] + \qty[\bigcap E_n]$; and the finiteness of $\nu(E_1)$ on the left hand side implies all the terms in the union converge absolutely. Therefore
    \begin{align*}
            \nu(E_1) - \nu\qty(\bigcap E_n) &= \lim_{n\to+\infty} \nu(G_n) = \lim_{n\to+\infty}\nu(E_1) - \nu(E_n)\\
            &= \nu(E_1) - \lim_{n\to+\infty}\nu(E_n),
    \end{align*}
    and cancelling terms finishes the proof.
\end{proof}

\end{document}