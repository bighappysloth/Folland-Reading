\documentclass[../../main.tex]{subfiles}

\begin{document}
\problem{4.13}
\begin{wts}
If $X$ is a topological space then $\bc{X}$ is a closed subspace of $B(X)$ in the uniform metric, and $BC(X)$ is complete.
\end{wts}
\begin{proof}
We will prove four things, the last two are just book-keeping. Parts (b, d) imply Part (c), as the closure of any set under a complete metric space is again complete.

\begin{enumalpha}
    \item $B(X)$ endowed with the uniform norm of an $f\in B(X)$
    \[
    \norm{f}_u = \sup \{|f(x)|,\: x\in X\}
    \]
    Is indeed a normed vector space.
    
    \item $B(X)$ with its norm (and induced metric), is a complete metric space. So that our $\{f_n\}\to f$ at worst, converges to $f\in B(X)$.

    \item If $\{f_n\}_{n\geq 1}\subseteq \bc{X}$ is a uniformly Cauchy sequence, and $f_n\to f$, then $f\in \bc{X}$.
    \item If $f$ is an adherent point of $\bc{X}$, then $f\in \bc{X}$.
\end{enumalpha}
    To show that $B(X)$ is a normed vector space, for any $k\in \mathbb{C}$, $f_1, f_2\in B(X)$, then at every $x\in X$
    \[
    |f_1(x) + kf_2(x)|\leq |f_1(x)| + |k|\cdot|f_2(x)|\leq \norm{f_1}_u + |k|\norm{f_2}_u
    \]
    And to show absolute homogeneity, note that $\sup{|kA|} = |k|\cdot\sup{A}$ for any non-empty bounded above set of reals $A$. This proves (a).\\
    
    To show (b), fix any Cauchy sequence in $B(X)$ (with respect to the uniform metric), then for every $\varepsilon>0$, there exists an $N$ so large that for every $n,m\geq N$ we have
    \[
    |f_n(x)-f_m(x)|\leq \norm{f_n-f_m}_u<\varepsilon
    \]
    This shows that $\{f_n(x)\}_{n\geq 1}\subseteq \mathbb{C}$ is Cauchy, and it makes sense to call its limit $f(x) = \lim f_n(x)$. To show that for this $f$,
    
    \begin{itemize}
        \item $f_n\to f$ uniformly, and
        \item $f\in B(X)$
    \end{itemize}

    Fix an $\varepsilon>0$, and there exists an $N$ so large that for every $m, n\geq N$ implies that
    \[
        \norm{f_n(x)-f_m(x)}_u < \varepsilon
    \]
    Since $\lim_{n\to\infty}f_n(x) = f(x)$, this means that
    \[
        \lim_{n\to\infty}|f_n(x)-f_m(x)|=|f(x)-f_m(x)|\leq\varepsilon
    \]
    The above holds for any $x$, hence
    \[
        \norm{f_m-f}_u\leq\varepsilon\implies \norm{f}_u \leq \norm{f_m - f}_u + \norm{f_m}_u<+\infty
    \]
    This proves both bullet points.\\
    
    Proof of (c): Now we will prove Theorem 4.13, for any sequence $\{f_n\}\subseteq \bc{X}$, if it does converge to some $f$ uniformly, then we claim that $f\in \bc{X}$. Note that $f\in B(X)$, so it suffices to show continuity at every $x_0\in X$.\\
    
    Fix any ball with radius $\varepsilon>0$ at $f(x_0)\in\mathbb{C}$, and since
    \begin{itemize}
        \item $\varepsilon/3>0$ induces some $N$ such that for every $n\geq N$, at every point $x\in X$
        \[
        |f_n(x_0)-f(x_0)|\leq\norm{f_n-f}_u<\varepsilon/3
        \]
        \item Another $\varepsilon/3$ gives us an open ball around $f_n(x_0)$ in $\mathbb{C}$ (using the same point $x_0\in X$). Continuity of $f_n$ gives us
        \[
        f_n^{-1}(\:B(\varepsilon/3,\: f_n(x_0))\:) = U\in\Tau_X
        \]
        \item If $x$ is a point in $U$,
        \[
            |f_n(x)-f(x)|\leq \norm{f_n-f}_i<\varepsilon/3
        \]
        this gives us the last $\varepsilon/3$.
    \end{itemize}
    Combining these three, 
    \[
    |f(x)-f(x_0)|\leq \underbrace{|f(x)-f_n(x)|+|f(x_0)-f_n(x_0)|}_{\text{uniform convergence}}+\underbrace{|f_n(x_0)-f_n(x)|}_{\text{continuity of }f_n}< \varepsilon
    \]
    So there exists some open set $U\in\Tau_X$ (and hence neighbourhood of every $x$), for every open ball of radius $\varepsilon>0$, around every $f(x)\in \mathbb{C}$, such that
    \[
    f(U)\subseteq B\in \Tau_\mathbb{C}
    \]
    Since the open balls are a neighbourhood base at every point in $\mathbb{C}$, and $f$ is continuous at every point $x\in X$, we must conclude that $f\in\bc{X}$.\\

    Part (d): Let $f\in \cl{\bc{X}}$. Notice $\bc{X}$ is a metric space, hence first countable. There exists a sequence $\{f_n\}\subseteq \bc{X}$ that converges to $f$. Convergent sequeneces in any metric space is Cauchy, apply Part (c) finishes the proof.
\end{proof}

\end{document}