\documentclass[../../main.tex]{subfiles}

\begin{document}
\subsection{Theorem 4.15}
\begin{wts}
Urysohn's Lemma. Let $X$ be a normal space, if $A$ and $B$ are disjoint closed subsets of $X$, then there exists a $f\in C(X,[0,1])$ such that $f=0$ on $A$ and $f=1$ on $B$.
\end{wts}
\begin{proof}
Let $r\in \Delta$ be as in Lemma 4.14, and set $U_r$ accordingly except for $U_1=X$. Define 
\[
f(x) = \inf\{r:x\in U_r\}
\]
Then for every $x\in A$ we have $f(x)=0$, since by the construction of the 'onion' function in Lemma 4.14, for each $r\in \Delta\cap(0,1)$, 
\[
x\in A\subseteq U_r\implies f(x)\leq r
\]
Since $r>0$ is arbitrary, we have $f(x)=0$. Now, for every $x\in B$, since $A$ and $B$ are disjoint, and $A\subseteq U_r\subseteq B^c$, then for every $x\in B$ means that $x$ is not a member of any $U_r$, but we set $U_1=X$. Since none of the $x\notin U_r$ for any $r\in(0,1)$, and $x\in U_1$, we get $f(x)=1$ on $B$.
\end{proof}
\end{document}