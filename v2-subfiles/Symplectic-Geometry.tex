\documentclass[../main-v2-manifolds.tex]{subfiles}
%%%%%%%%%
%%%%%%%%%
%%%%%%%%%
\makeatletter
\newcommand*{\addFileDependency}[1]{% argument=file name and extension
\typeout{(#1)}% latexmk will find this if $recorder=0
% however, in that case, it will ignore #1 if it is a .aux or 
% .pdf file etc and it exists! If it doesn't exist, it will appear 
% in the list of dependents regardless)
%
% Write the following if you want it to appear in \listfiles 
% --- although not really necessary and latexmk doesn't use this
%
\@addtofilelist{#1}
%
% latexmk will find this message if #1 doesn't exist (yet)
\IfFileExists{#1}{}{\typeout{No file #1.}}
}\makeatother

\newcommand*{\myexternaldocument}[1]{%
\externaldocument{#1}%
\addFileDependency{#1.tex}%
\addFileDependency{#1.aux}%
}
%------------End of helper code--------------
%%%%%%%%
% PUT ALL EXTERNAL DOCUMENTS YOU WANT TO REFERENCE IN THIS SECTION
\myexternaldocument{./Preliminaries} % Reference the Preliminiaries Page
%%%%%%%%%
%%%%%%%%%
%%%%%%%%%
%%%%%%%%%
%%%%%
%%%%%
% filter out everything but the definitios and the theorems.
%\usepackage{xcomment}
%\xcomment{definition,wts,lemma,corollary,note}
%%%%%%%
%%%%%%%
\begin{document}
%%%%%%%%
%%%%%%%%
\graphicspath{{../images/}{images/}} 
%%%%%%%%
%%%%%%%%
%%%%% Custom symbols for symplectic capacities
\providecommand{\Grom}{\mathrm{Gromov}}
\providecommand{\Periodic}{\mathrm{Periodic}}
\providecommand{\frakc}{\mathcal{C}}
\providecommand{\frakco}{\mathcal{C}_0}
\providecommand{\hcal}{\mathcal{H}}
\providecommand{\hcala}{\mathcal{H}_a}
\providecommand{\osc}{\mathrm{osc}}
%%%%%%%
%%%%%%%
\fchapter{6: Symplectic Geometry}
% Let $n\geq 1$. We will be working in $\real^{2n}$. Let us define
% \begin{itemize}
%     \item $X = C^\infty(S^1\,\realtn)$, where $S^1 = \{z\in\complex, \abs{z}=1\}$.
%     \item Recall the Sobolev space $M = H_{1/2}$ is equipped with the inner product and norm defined in \cref{eq:hofer 1/2 inner product,eq:hofer 1/2 norm}.
% \end{itemize}
% \[
%     s
% \]
% % %%%%
% Loops are always closed, and we assume all loops are $1$-periodic. We will not distinguish between $S^1$ and $\real/\mathbb{Q}$. The measure we will use on $S^1$ is the Lebesgue measure.\\
% %%%
% \topheader{Almost complex structure}
% \begin{definition}[Almost complex structure]
%     A smooth manifold $M$ of dimension $2n$ has an \emph{almost complex structure} whenever it admits a smooth 'transformation' on the tangent spaces of each $x\in M$, 
%     \[
%     J_x: T_xM \to T_xM\quad J_xJ_x = -\id{T_x{M}}
%     \]
% \end{definition}
% \begin{remark}
%     This transformation should be interpreted in the 'tensor' sense. If $n = 1$, then $J$ is a section of the $\Tau^2 TM$ bundle, so that $J_x\in L(T_xM, T_xM;\real)$ is a $2$-covariant tensor --- which can be identified as a bilinear form on each tangent space.
% \end{remark}
\topheader{Primer on Differential Forms}
\begin{remark}[Finite-dimensional Manifolds]
    We assume all manifolds modelled over $\realn$ ($n\geq 1$) are of class $C^\infty$, and are equipped with Hausdorff, second-countable topologies.
\end{remark}
Let $M$ be a manifold modelled on $\realn$.
\begin{itemize}
    \item $\vField(M)$ = ($C^\infty(M)$) module of vector fields on $M$,
    \item $\cvField(M)$ = module of covector fields on $M$,
    \item $\Tau^{(j,k)}(M)$ = module of $j$-contravariant, $k$-covariant tensor fields on $M$.
    \item $\Tau^{k}(M)$ = $\Tau^{(0,k)}(M)$.
    \item $\Omega^k(M)$ = module of $k$-forms on $M$.
\end{itemize}

\begin{note}[Covariant and Contravariant Tensors]
    We recall that if $V$ is a $\real$-vector space, a $j$-contravariant, $k$-covariant tensor on $V$ --- denoted by $F$ --- is a $(j+k)$ linear mapping that takes $j$-covectors, and $k$-vectors to a real number. In symbols,
    \[
        F: (V^*)^j \times V^k\to\real\quad\text{is multilinear.}
    \]
    We denote the space of $(j,k)$ tensors on $V$ by $\Tau^{(j,k)}(V)$. The space of $(0,0)$ tensors on $V$ is identified with $\real$ --- as it depends on $0$ arguments. \\
    
    If $V$ is finite dimensional, then $V=\Tau^{(1,0)}(V)$, and $V^* = \Tau^{(0,1)}(V)$.
    Similarly, $\vField(M) = \Tau^{(1,0)}(M)$,  $\cvField(M) = \Tau^{(0,1)}(M)$ and $C^\infty(M) = \Tau^{(0,0)}(M)$.
\end{note}
If $N$ is another manifold and $u: M\to N$ a morphism, because the differential of $u$ pushes tangent vectors from $TM$ into $TN$, we identify $du$ with the mapping that pushes, where
\[
    du: \prod_{\underline{k}}TM\to \prod_{\underline{k}}TN,\qqtext{and}du(p)[v_{\underline{k}}] = (du(p)[v_{1}],\ldots, du(p)[v_k]).
\]
With $u: M\to N$ still being a morphism, 
\begin{itemize}
    \item for every $f\in C^\infty(N)$, the \emph{pullback} through $u$ is the precomposition $u^*f = f\circ u\in C^\infty(M)$, and
    \item for every $A\in\cvField(N)$, the \emph{tensor field pullback} through $u$ is the precomposition.  It is defined by
    \[
        (u^*A)(p)(v) = A[u(p)][du(p)(v)],\quad\text{where the square brackets are for readability.}
    \]
    For a general $A\in \Tau^{k}N$, we have
    \[
        (u^*A)(p)(v_{\underline{k}}) = A[u(p)][du(p)(v_{\underline{k}})],\quad\text{for an arbitrary }p\in M,\: v_{\underline{}k}\in T_pM.
    \]
    \item If $u$ is a diffeomorphism, we define the \emph{vector field pullback} of a vector field $Y\in\vField(N)$ by
    \[
        (u^*Y)(p) = du^{-1}\qty(Y_{u(p)}) = (du^{-1}\circ Y\circ u)(p).
    \]
\end{itemize}
We recall a few facts from differential geometry.
\begin{itemize}
    \item If $f\in C^\infty(M)$, the \emph{exterior derivative} of $f$ is the covector field $df$ with coordinate representation $df(p) = \partial_i f(p) dx^i$.
    \item If $A\in \Omega^k(M)$, the \emph{exterior derivative} of $A$ is a $k+1$ form that is defined by its local coordinate representation. 
    \item The exterior derivative $d$ commutes with the tensor field pullback. That is, for every $A\in \Omega^k(N)$, $u^*(dA) = du^*A$.
\end{itemize}
\begin{definition}[Exterior Derivative in Local Coordinates]
    Let $M$ be a manifold modelled on $\realn$, and $A\in \Omega^k(M)$. If $(x^i)$ are the local coordinates in some open subset $U\osub M$, $A$ can be written as the tensor product of dual basis vectors $(dx^i)$.
    \begin{equation}
        A = \isum_{J}A_J dx^J
        \label{eq:k form isum local representation}
    \end{equation}
    where $\isum$ refers to an increasing sum taken over $k$-indices. We define the \emph{exterior derivative of $A$} by the $k+1$ form in local coordinates
    \begin{equation}
        dA = d\qty(\isum_J A_J dx^J) = \isum_J dA_J \wedge dx^J.
        \label{eq:exterior derivative of isum local representation 1}
    \end{equation}
    Unboxing the differential of $A_J$ and the wedge product, \cref{eq:exterior derivative of isum local representation 1} becomes:
    \begin{equation}
        dA = \isum_J \sum_{i=\underline{n}}\pdv{A_J}{x^i}dx^i\wedge dx^J = \isum_J\sum_{i=\underline{n}} \pdv{A_J}{x^i}dx^{(i,J)}.
        \label{eq:exterior derivative of isum local representation 2}
    \end{equation}
\end{definition}
\begin{example}[Exterior Derivative in Coordinates]\label{exmp:exterior derivative}
    Let $M = \real^3\setminus \{0\}$, we will use the standard coordinates $(x,y,z)$ on $M$. 
    \begin{enumerate}
        \item $f(x,y,z) = (x^2 + y^2)^{1/2}$ = scalar valued function.
        \item $A(x,y,z) = (y-x)dz - zdy$ = covector field.
        \item $B(x,y,z) = f(x,y,z)dx\wedge dy$ = $2$-form.
    \end{enumerate}
    Exterior Derivative of $f$:
    \[
        df(x,y,z) = \pdv{f}{x}dx + \pdv{f}{y}dy + \pdv{f}{z}dz = \frac{xdx + ydy}{f(x,y,z)}
    \]
    Exterior Derivative of $A$:
    \begin{align*}
        dA(x,y,z) &= d(y-x)\wedge dz + d(-z)\wedge dy \\[1ex]
        &= dy\wedge dz - dx\wedge dy - dz\wedge dy = 2dy\wedge dz -dx\wedge dy
    \end{align*}
    Exterior Derivative of $B$:
    \[
        dB(x,y,z)= df(x,y,z)\wedge (dx\wedge dy) = \frac{xdx + ydy}{f(x,y,z)}\wedge (dx\wedge dy)=0
    \]
\end{example}
\begin{remark}[Exterior Derivative on Banach Manifolds]
    If $X$ is a Banach space, which is also a Banach manifold of class $C^k$ for $k\geq 1$, the exterior derivative of $C^k$ function $f$ is a $C^{k-1}$ covector field whose evaluation at $p\in X$ coincides with the Frechet Derivative $Df(p)$. Recall that $Df(p)$ is the unique linear map that satisfies
    \[
        f(p + v) = f(p) + Df(p)(v) + o(\abs{v}).
    \]
\end{remark}
\begin{remark}[Closed, and exact differential forms]
    Let $A\in \Omega^k(M)$ be a $k$-form on manifold $M$. 
    \begin{itemize}
        \item It is \emph{closed} whenever $dA = 0$, and
        \item is \emph{exact} whenever $A = dB$ where $B \in \Omega^{k-1}(M)$.
    \end{itemize}
\end{remark}
\begin{remark}[Poincare's Lemma]
    A subset $S\subseteq\realn$ is said to be \emph{star-shaped} if there exists some $a\in S$ where $\{a + (b-a)[0,1]\}\subseteq S$ for every $b\in S$. That is, the straight line segment between $a$ and every point $S$ is contained in $S$.\\

    Poincare's Lemma states that, if $U$ is an open, star-shaped subset of $\realn$, then every closed form is exact.
\end{remark}
\begin{remark}[Line integral]
    Let $\gamma: [0,L]\to M$ where $M$ is a smooth manifold. For any smooth $1$-form $\lambda$ on $M$, the integral of $\gamma$ over $\lambda$ is the integral
    \[
        \int_\gamma \lambda =  \int_{[0,L]} \gamma^*\lambda = \int_{0}^{L}\lambda(\gamma(t))(\mathring{\gamma}(t))dt.
    \]
\end{remark}
\begin{example}[Line integral in coordinates]
    Let $\gamma(t) = (\cos(2\pi t), \sin(2\pi t), 0)$ for $t\in[0,1]$, and the covector field 
    \[A(x,y,z) = \frac{xdy - ydx}{x^2 + y^2}\quad\forall (x,y)\neq 0.\]
    Suppressing the trigonometric arguments, the line integral of $A$ over $\gamma$ is given by
    \[\int_{\gamma}A = \int_{0}^{1} A[\gamma(t)][\mathring{\gamma}(t)]dt = \int_0^1 \frac{\cos dy - \sin dx}{\cos^2 + \sin^2}(2\pi (-\sin, \cos,0))dt\]
    which gives
    \[\int_{\gamma} A = 2\pi\int_0^1\cos\cos - \sin(-\sin) dt = 2\pi.\]
\end{example}
%
%% Footnote: a differential form is said to be closed whenever its exterior derivative is zero.
%% Footnote: a differential form is said to be exact whenever it is the exterior derivative of another differential form.
%% Poincare's Lemma: if M is a star-shaped domain, then a differential 1-form is closed if and only if it is exact.
%% Fact: Exact forms are always closed, but closed forms are not always exact - depends on deRham cohomology.
\topheader{Standard Symplectic Form}
We begin the case in $\real^2$. The \emph{standard symplectic form} on $\real^2$ is the bilinear form represented by the matrix (with respect to the standard basis) in \cref{eq:std symplectic form r2}.
\begin{equation}
    J_2 = \mqty[0 & 1\\ -1 & 0]
    \label{eq:std symplectic form r2}
\end{equation}
The following note summarizes several properties of $J$.
\begin{note}[Properties of the standard symplectic form on $\real^2$]\label{note:properties of std symplectic form r2}
    If $x,y\in\real^2$, \cref{eq:std symplectic form r2} defines a pairing $\omega_0\in \Omega^2(\real^2)$ between $x$ and $y$. Where $\omega_0(x,y) = \langle x, Jy\rangle_{\real^{2}}$. An easy computation in coordinates will show that
    \begin{equation}
        \omega_0(x,y) = \begin{bmatrix}x_1 & x_2
        \end{bmatrix}\begin{bmatrix}0 & 1 \\ -1 & 0\end{bmatrix}\begin{bmatrix} y_1 \\ y_2\end{bmatrix} = \det\qty(\begin{bmatrix} x_1 & y_1 \\ x_2 & y_2\end{bmatrix}) = \det(x,y)
        \label{eq:std symplectic action pw r2}
    \end{equation}
    Furthermore, 
    \begin{itemize}
        \item $J$ is non-singular and skew-symmetric, and $J^{-1} = (-1)J$.
        \item $\omega$ is non-singular and skew-symmetric, it is a non-degenerate $2$-form on $\real^{2n}$ by \cref{lem:characterisation of bilinear forms}.
        \item Left multiplication by a vector $v = (v^1, v^2)$ reads 
        and $\omega_0(v,\cdot) = v^1\varepsilon^{2} + (-1)v^2\varepsilon^1$.
        \item Right multiplication by $v$: by skew-symmetry of $J$ reads: $\omega_0(\cdot, v) = (-1)v^1\varepsilon^2 + v^2\varepsilon^1$.
    \end{itemize}    
\end{note}

\begin{definition}[Standard symplectic form]\label{def:std-symplectic-form}
    Let $n\geq 1$, the \emph{standard symplectic form} is the bilinear form defined by the matrix representation in \cref{eq:std symplectic form r2n 2}.
    \begin{equation}
        J =J_2\otimes \id{\realn}=\begin{bmatrix}
            \admat[0]{I_n,-I_n}
        \end{bmatrix}\quad\text{where }\otimes\text{ denotes the Kronecker Product}.
        \label{eq:std symplectic form r2n 2}
    \end{equation}
    The matrix in \cref{eq:std symplectic form r2n 2} induces a bilinear pairing, which we will denote by $\omega_0\in \Omega^2(\realtn)$. Its defining property is that it computes the sum of $n$ $2\times 2$ determinants, as shown in \cref{eq:std symplectic form r2n determinants}. 
    \begin{equation}
        \omega_0(x,y) =\langle x,y\rangle_{\omega_0} = \sum_{i=\underline{n}}\det\qty(\mqty[x^i & y^i \\ x^{n+i} & y^{n+i}])
        \label{eq:std symplectic form r2n determinants}
    \end{equation}
    We can rewrite $\omega_0$ using the language of differential forms:
    \begin{equation}
        \omega_0 = \sum_{i=\underline{n}}\varepsilon^{i}\wedge\varepsilon^{n+i}.
        \label{eq: std symplectic form r2n wedge products using covectors}
    \end{equation}
\end{definition}
The properties outlined in \cref{note:properties of std symplectic form r2} all hold for $\realtn$. Moreover, $\omega_0$ is exact, as one can verify that if $\lambda = \sum x^i dx^{n+i}$ with the sum taken over $\underline{n}$, then $d\lambda = \omega_0$. Recall if $p,v\in\realtn$, 
\[
    \lambda(p) = \sum p^{i}dx^{n+i}\qqtext{and} \lambda(p)(v) = \sum p^{i}v^{n+i}.
\]
%
\begin{remark}[Alternate Symplectic Structure]
    Some texts use \cref{eq:std symplectic form r2n 1}, or $J_{2n} = \mqty[0 & -I_n\\ I_n & 0]$. The following decomposition is called the \emph{maximal hyperbolic decomposition} of $\realtn$, see \cite{Roman2007Advanced} Chapter 13. We will return to this later when we discuss periodic solutions on ellipsoids.
    \begin{equation}
    \wig{J}_{2n} = \id{\realn}\otimes J_{2} = \begin{bmatrix}
        \dmat{\symatrix,\ddots,\ddots,\symatrix}
    \end{bmatrix}
    \label{eq:std symplectic form r2n 1}
\end{equation}
\end{remark}
%
%
\begin{note}[Computations with the standard symplectic form]\label{note: symplectic form summation computations}
    We call the standard symplectic form given in \cref{eq:std symplectic form r2n 2} in terms of the Kronecker delta. A moment's thought will show that $J = [\delta_{(i,j-n)} - \delta_{(i,j+n)}]_{ij} = [\delta_{(n+i,j)} - \delta_{(i-n,j)}]_{ij}$. Left multiplication by a vector $v = (v^{\underline{2n}})$ yields
    \begin{align*}
        \omega_0(v,\cdot) &= \langle v, J\cdot\rangle_{\realtn} = v^i[\delta_{(i,j-n)} - \delta_{(i,j+n)}]\varepsilon^j\\[2ex]
        &=\sum_{i=\underline{2n}}v^i\varepsilon^{i+n} - v^i\varepsilon^{i-n} =\sum_{i=\underline{n}}v^i\varepsilon^{i+n} - v^{i+n}\varepsilon^{i}
    \end{align*}
    Right multiplication then give us 
    \[
        \omega_0(\cdot, v) = \langle \cdot,Jv\rangle_{\realtn} = \sum_{i=\underline{n}}(-1)v^i\varepsilon^{i+n} + v^{i+n}\varepsilon^i.
    \]
\end{note}
\topheader{Symplectic Manifolds}
In this section, we introduce a differential geometric viewpoint, allowing ourselves to work with arbitrary symplectic structures.
\begin{definition}[Symplectic Manifold]\label{def:symplectic manifold}
    A \emph{symplectic manifold} is a manifold $M$ modelled on $\realtn$ (for $n\geq 1$), equipped with a \textbf{closed, non-degenerate $2$-form $\omega$}. We sometimes refer the tuple $(M,\omega)$ as the \emph{symplectic structure}.
\end{definition}
\begin{definition}[Symplectomorphism]
    Let $(M,\omega)$ and $(N,\eta)$ be symplectic manifolds of dimension $2m$ and $2n$ respectively. A mapping $u: M\to N$ is a \emph{symplectomorphism} (or is symplectic as an adjective) whenever it preserves the symplectic structure under the tensor pullback. That is,
    \[
        u^*\eta =\omega,\qqtext{which means} \omega(p)(v_1,v_2) = \eta(u(p))\biggl(du(p)[v_1,v_2]\biggr),
    \]
    for every $p\in M$, and $v_{\underline{2}}\in T_p M$. An embedding that is a symplectomorphism is called a \emph{symplectic embedding}.
\end{definition}
\begin{example}[Symplectomorphism on $\realtn$]
Let $\varphi:\realtn\to\realtn$ be smooth, we say $\varphi$ is a \emph{symplectomorphism} (or $\varphi$ is symplectic as an adjective) whenever it preserves $\omega$. That is,
\[
    \left\langle D\varphi(x)(v_1), D\varphi(x)(v_2)\right\rangle_{\omega_0} = \omega_0(v_1, v_2),\forall x,v_1,v_2\in\realtn,
\]
where $D\varphi(x)$ refers to the Jacobian matrix of $\varphi$ evaluated at $x\in\realtn$. If $\varphi$ is a $C^\infty$ diffeomorphism and $\varphi$ and its inverse are symplectomorphisms, we call $\varphi$ a \emph{symplectic diffeomorphism} or a \emph{symplectic isomorphism}.
\end{example}
\begin{definition}[Symplectic action on $(M,\omega)$]
    If $(M,\omega)$ is a symplectic manifold, we write
    \[
        \langle v_{1}, v_{2}\rangle_{\omega(p)} = \omega(p)(v_1, v_2),\quad\text{for every }p\in M,\text{ and } v_{1},v_2\in T_pM.
    \]
    Given an interval $\mathcal{I}\subseteq\real$, the \emph{symplectic pairing} between two curves is defined to be $A(\gamma,\eta) = 2^{-1}\int_{\mathcal{I}} \langle \mathring{\gamma},\eta\rangle_{\omega_0}$, for every $\gamma,\eta\in C^\infty(\mathcal{I}, M)$. The \emph{symplectic action} on a curve $\gamma$ is 
    \[
        A(\gamma) = A(\gamma,\gamma) =2^{-1}\int_{\mathcal{I}}\langle\mathring{\gamma},\gamma\rangle_{\omega}.
    \]
\end{definition}
%
%
%
% \begin{note}[Alternate interpretation using tensor pullbacks]
%     Equivalently, we say $\varphi$ is a symplectomorphism whenever $\omega_0$ is invariant under its tensor pullback. This means, 
%     \begin{equation}
%         \varphi^*(\omega_0) =\omega_0
%         \label{eq:symplectomorphism tensor pullback}
%     \end{equation}
%     This is the same as
%     \[
%         (D\varphi(x))^{T}J D\varphi(x) = J\quad\text{in the sense of matrix multiplication.}
%     \]
%     An immediate consequence of the previous formula is that a symplectomorphism $\varphi$ must be volume-preserving, meaning $\det D\varphi(x) = 1$ at every point.
% \end{note}
\begin{remark}[Symplectomorphisms are volume-preserving]
    If $\varphi:M\to N$ is a symplectomorphism, then the determinant of the Jacobian matrix (with respect to any pair of charts) is $1$. This means, if $\varphi:M\hookrightarrow\to N$ is a symplectic embedding, then $\mathrm{vol}(M)\leq \mathrm{vol}(N)$, where $\mathrm{vol}$ refers to the Riemannian Volume.
\end{remark}
%
\begin{lemma}[Symplectic invariance of $A$ on $\realtn$]\label{lem:symplectomorphisms from realtn preserve symplectic action}
    If $\varphi:(\realtn,\omega_0)\to (M,\eta)$, then $A(\varphi\circ\gamma) = A(\gamma)$ for every $\gamma\in \Omega$.
\end{lemma}
\begin{proof}
    Because the exterior derivative commutes with the tensor pullback, we have
    \[
        d(\lambda - \varphi^*\lambda) = d\lambda - \varphi^*(d\lambda)= 0
    \]
    whence $\lambda - \varphi^*\lambda$ is a closed differential form. We see that
    \[
        A(\varphi\circ\gamma) - A(\gamma) = \int_{\gamma}\varphi^*\lambda - \int_{\gamma}\lambda = \int_{\gamma}(\varphi^*\lambda - \lambda)
    \]
    It follows from Poincare's Lemma that the right hand vanishes, since it is the closed curve over an exact form.
\end{proof}
A simple modification of the proof then yields:
\begin{corollary}[Symplectic invariance of $A$, when $H_{dR}^1=0$]
    If $\varphi: (M,\omega)\to (N,\eta)$ is a symplectomorphism, and $H_{dR}^{1}(M) = 0$, then $A(\varphi\circ\gamma) = A(\gamma)$ for every curve $\gamma$ in $M$.
\end{corollary}
\topheader{Hamiltonian Vector Fields}
We begin with some well known sign conventions for area.
\begin{quote}
    Fix any two vectors $v_1, v_2\in \realn$, we say that positive area opens to the left, or anti-clockwisely from $v_1$.
\end{quote}
Because of this, we refer to $\det(v_1, v_2)$ as the \emph{area spanned from $v_1$ to $v_2$} if $v_1,v_2\in\real^2$, and we call $\omega_0(v_1,v_2)$ the \emph{symplectic area from $v_1$ to $v_2$}. Moreoever, if $S$ is a compact region in $\realn$ whose (topological or manifold) boundary $\partial S$ can be traversed by a continuous curve $\gamma:[0,1]\to\realn$. 
\begin{quote}
We say $\gamma$ is positively oriented (with respect to the \emph{Stokes' orientation}), whenever the region $S$ lies to the left of $\gamma$ at every point.
\end{quote}
\begin{figure}[h!]
    \centering
    \includegraphics[width=0.5\linewidth]{images/positive-area-opens-to-left-sketch.png}
    \caption{Illustrations of area sign conventions}
    \label{fig:area sign conventions}
\end{figure}
\begin{remark}[Positive Gradient Flow]\label{rmk:gradient flow}
Let $H\in C^\infty(\realtn)$, the \emph{positive gradient flow} of $H$ is the vector field $\nabla H$ such that at every point $p\in \realtn$, and $v_p\in T_p \realtn$:
\begin{quote}
    The \textbf{angle between $\nabla{H}(p)$ and $v_p$ is equal to $DH(p)(v_p)$}, where
    \begin{equation}
    H(p+v_p) = H(p) + DH(p)(v_p) + o(\abs{v_p})\quad\text{for sufficiently small }v.
    \label{eq:Frechet Derivative as best possible linear approximation}
\end{equation}
\end{quote}
By the 'angle' we refer to the Euclidean inner product which takes on values in $\real$ instead of in $[-\pi, +\pi]$. Moreover, the \emph{Euclidean gradient} of $H$ in coordinates is given by
\[\nabla H = (\partial_{\underline{2n}}H)\in\vField(\realtn).\]
\end{remark}
\begin{definition}[Hamiltonian Flow]
    The \emph{Hamiltonian flow} of $H$ is the vector field $X_H$ such that at every point $p\in \realtn$, and $v_p\in T_p\realtn$:
    \begin{quote}
        The \textbf{symplectic area from $X_H(p)$ to $v_p$ is equal to $DH(p)(v_p)$.}
    \end{quote}
    More precisely, the Hamiltonian flow of $H$ is defined by the sharpening the covector field of $H$:  $X_H = \omega_0^{\wedge}(dH)$, such that
    \begin{equation}
        \omega_0(X_H(p), v_p) = dH(p)(v_p)\quad\text{for all }p\in \realtn,\: v_p\in T_p\realtn.
        \label{eq:hamiltonian flow definition symplectic pairing}
    \end{equation}
\end{definition}
In Euclidean space, $X_H$ has a simple structure, and is related to the $\nabla H$ by a factor of $J$. 
\begin{lemma}[Hamiltonian Flows in Euclidean Space]\label{lem:hvf in euclidean space formula}
    The Hamiltonian flow of $H\in C^\infty(\realtn,\real)$, $X_H$ has matrix representation which satisfies
    \begin{equation}
        X_H = J\nabla H.
        \label{eq: hvf in euclidean space formula}
    \end{equation}
\end{lemma}
\begin{proof}
    Let $p$ and $v_p$ be arbitrary, it follows from the definition of $X_H$ that
    \[
        \langle X_H(p), v_p\rangle_{\omega_0} = \langle X_H(p), Jv_p\rangle_{\realtn} = dH(p)(v_p) = \langle\nabla H(p), v_p\rangle_{\realtn}.
    \]
    Notice that $J$ is skew-symmetric, so we can move $J$ over to the other side of the bracket at the cost of a minus sign, hence:
    \[
        \langle (-1)J X_H(p), v_p\rangle_{\realtn} = \langle \nabla H(p),v_p\rangle_{\realtn}.
    \]
    The proof is complete upon seeing that $(-1)J=J^{-1}$.
\end{proof}
\topheader{Statement of the Weinstein's Conjecture}
\begin{definition}[Closed submanifold]\label{def:closed-submanifolds}
    Let $M$ be a manifold modelled on $\realn$. A submanifold $S\subseteq M$ is said to be \emph{closed} whenever it is a compact subset of $M$, and the $S$ is a manifold without boundary.
\end{definition}
\begin{remark}[Stokes' Theorem]
    Let $M$ be a manifold with boundary modelled on $\realtn$, for any compactly supported $(n-1)$ form $\omega$:
    \begin{quote}
        The integral of $d\omega$ over $M$ is equal to the integral of $\omega$ over $\partial M$. In symbols,
        \[\int_{M}d\omega = \int_{\partial M}\omega.\]
    \end{quote}
    If $S$ is a closed submanifold of $M$, and $\omega$ a $(n-1)$-form, an immediate corollary is that
    \[\int_S d\omega = \int_{\partial S}\omega = 0.\]
\end{remark}
\begin{definition}[Regular hypersurface]\label{def:regular hypersurface}
    A \emph{regular hypersurface} on a smooth manifold $M$ is a subset $S = f^{-1}(c)$ where $f\in C^\infty(M,\real)$, and $df(p)\neq 0$ for every $p\in S$. We call $f$ the \emph{defining function} of $S$ which admits a natural manifold structure that makes $S$ a submanifold of $M$. 
\end{definition}
\begin{definition}[Energy surface]\label{def:energy surface}
    An \emph{energy surface} is a compact, regular hypersurface of a symplectic manifold $(M,\omega)$.
\end{definition}
\begin{remark}[Terminology surrounding Weinstein' Conjecture]
    Let $X$ be a vector field on a manifold $M$.
\begin{itemize}
    \item A \emph{solution} to $X$ is a mapping $\gamma: \mathcal{I}\to M$ where $\mathring{\gamma}(t) = X(\gamma(t))$ at every $t$ in the open interval $\mathcal{I}$.
    \item An \emph{orbit} of $X$ is a non-constant solution.
\end{itemize}
Let $(M,\omega)$ be a symplectic manifold, and $S$ an energy surface.
\begin{itemize}
    \item A \emph{smooth defining function} of $S$ is a function $F\in C^\infty(M,\real)$ such that $S$ is a regular level set of $F$. We sometimes say $F$ is a \emph{defining function} of $S$ when it is implicit.
    \item A \emph{characteristic} of $S$ is the image $\gamma(\mathcal{I})$ where $\gamma$ is a solution of the Hamiltonian flow of a defining function of $S$. 
    \item Given a characteristic of $S$, $\gamma(\mathcal{I})$. It is \emph{closed} whenever $\gamma$ is periodic, and is \emph{non-degenerate} whenever $\gamma$ is an orbit.
\end{itemize}

\end{remark}

We conclude this section by stating Weinstein's conjecture on $\realtn$ and proving the first reduction.
\begin{quote}
    Does every energy surface on $(\realtn,\omega_0)$ admit a periodic orbit?
\end{quote}
A more abstract reformulation of the conjecture is given below.
\begin{quote}
    Given an energy surface $S$, does its line bundle $\mathcal{L}(S)=\{(x,v)\in TS,\:  v\in \rad{\omega_0(p)}\}$ admit a non-degenerate closed characteristic?
\end{quote}

\begin{wts}[WC Reduction 1 --- Independence of Hamiltonian]\label{thm:wc reduction 1 hamiltonian}
    Let $S$ be a compact, regular hypersurface on a symplectic manifold $(M,\omega_0)$. If $F, G\in C^\infty(M)$ are defining functions of $S$ such that
    \[S = F^{-1}(c) = G^{-1}(c'),\]
    where 
    \[dF(x)\neq 0\qqtext{and}dG(x)\neq 0\quad\forall p\in S.\]
    Then, there exists a $\rho\in C_c^\infty(M,\real)$ such that for every $x\in S$, $\rho(x)\neq 0$ and $dF(x) = \rho(x) dG(x)$, and $X_F(x) = \rho(x) X_G(x)$. \\

    Assuming the existence of such a $\rho$, 
    \begin{itemize}
        \item For any $x\in S$, let $\varphi_x(s) = \varphi(s,x)$ and $\theta_x(t) = \theta(t,x)$ denote the integral curves starting at $x$ of $X_F$ and $X_G$. The smooth function $\alpha$ constructed by solving the IVP in \cref{eq:wc reduction1: alpha reparam} relates the two flows by its reparameterization.
        \begin{equation}
            \dv{\alpha}{s} = \rho(\varphi_x(s))\quad\alpha(0)=0
            \label{eq:wc reduction1: alpha reparam}
        \end{equation}
        By reparameterization we mean that $\varphi_x(s) = \theta_x(\alpha(s))$ for all $s$ whenever either side is defined.
        \item The periodic orbits of $X_{F}$ and $X_G$ on $S$ correspond bijectively.
        \item For any $x\in S$, $\varphi_x$ is a non-degenerate periodic orbit if and only if $\theta_x\circ\alpha$ is.
    \end{itemize}
\end{wts}
\begin{proof}
    Both $F$ and $G$ are global defining functions of the submanifold $S$, if $p\in S$ is arbitrary, the exterior tangent space coincides precisely with $\Ker dF(p) = \Ker dG(p) = T^{\mathrm{ext}}_{p}(S)$. (Lee 5.38, 5.40). Since $T^{\mathrm{ext}}_{p}S$ has dimension $1$, there exists a suitably chosen coordinate chart $(U, \zeta)$ about $p$ such that $dz\in\cvField(\real)$ spans the coordinate representation of $\cvField(T_p(S))$, and there exists smooth functions $u_F$ and $u_G$ where
    \[
        \zeta(dF(q)) = u_F(q)dz\qqtext{and}\zeta(dG(q)) = u_G(q)dz\quad\text{locally.}
    \]
    This uniquely defines $\rho$ on a neighbourhood of $p$ (by definition of the abstract tangent space), we can assume $\rho$ is compactly supported by appealing to Urysohn's Lemma for smooth manifolds.\\

    The symplectic form is $C^\infty(M)$-linear, hence $X_F = \rho X_G$ on a precompact neighbourhood of $S$. Given a point $x\in S$, we see that
    \[
        \varphi_x(s) = X_F(\varphi_x(s)) = \rho(\varphi_x(s)) X_G(\varphi_x(s)).
    \]
    Using \cref{eq:wc reduction1: alpha reparam}, we can define a smooth function $\alpha$ (because $\rho$ is smooth). Using the chain rule, and suppressing $\varphi_{x}(s)$:
    \[
        \dv{s} \theta_{x}(\alpha(s))\eval_s = \rho X_G = X_F
    \]
    So that $\theta_{x}\circ \alpha$ is an integral curve of $X_F$ starting at $x$, and must be equal to $\varphi_{x}$ by uniqueness. Next, 
    \begin{itemize}
        \item if $\theta_{x}(\alpha(s))$ is a periodic orbit of $X_G$, it follows that $\varphi_x(s)$ is a periodic orbit of $X_F$; and
        \item because $\rho$ is either strictly positive or negative, $\varphi_{x}(s)$ is a critical point of $X_F$ iff $\theta_{x}(\alpha(s))$ is a critical point of $X_G$.
    \end{itemize}
    At last, if $X_F = \rho X_G$ about $S$, then $X_G = \rho^{-1} X_F$. Let $\beta(x,t) = \int_0^t \rho(x,u)^{-1}du$, and we obtain
    \[
        \dv{t}\varphi_{x}(\beta(t))\eval_t = \rho^{-1}X_F = X_G,
    \]
    and rehearsing the same argument we had for $\alpha$ completes the proof.
\end{proof}

%
%
%
% \topheader{Symplectic Manifold}
% Let $n\geq 1$ be an integer, a \emph{symplectic manifold} of class $C^\infty$ is a real manifold of dimension $2n$ that is equipped with a $2$-form $\omega\in \Omega^2(M)$ such that
% \begin{itemize}
%     \item $\omega$ is non-degenerate, that is: at every $x\in M$ for every tangent vector $v_1\in T_xM$ there exists another tangent vector $v_2\in T_xM$ such that $\omega(v_1,v_2)$. Alternatively, the flat map is non-singular --- meaning
%     \[
%         \breve{\omega}(v_1) = \omega(v_1,\cdot) \neq 0\in T^*_xM\quad\forall v_1\in T_xM
%     \]
%     \item $\omega$ is closed, meaning $d\omega=0$ where $d$ refers to the exterior derivative of $\omega$.
% \end{itemize}
% We call a $2$-form that satisfies the two properties above a \emph{symplectic form} on $M$. 
% \begin{remark}
%     All finite-dimensional manifolds hereinafter will be assumed $C^\infty$.
% \end{remark}
%
%
%
\topheader{Symplectic Action}
We return to a more abstract-analytic perspective. Let $(X,\mcal,\mu)$ be a measure space, suppose $\gamma, \eta: X\to \realtn$ is an $L^2$ function, in the sense that it is $L^2$ in each coordinate. Holder's inequality tells us that
\[
    2^{-1}\langle \gamma,\eta\rangle_{\omega_0} = 2^{-1}\int_{X}\langle \gamma(x),\eta (x)\rangle_{\omega_0} d\mu(x)\quad\text{converges absolutely.}
\]
With this, we can extend the symplectic form $\omega_0$ to $L^2$ mappings into $\realtn$. The following is a natural function space to consider.
\begin{definition}[Loop Space]
    We define the space of \emph{loops} as the function space $\Omega = C^\infty(S^1, \realtn)$. It is equipped with the \emph{symplectic pairing}, which is denoted by $A:\Omega\times\Omega\to\real$;and $A(\cdot,\cdot)$ is defined by the integral in \cref{eq:loop space S1 r2n symplectic pairing}.
    \begin{equation}
        A(x,y) = 2^{-1}\int_{S^1} \langle \mathring{x}(t), y(t) \rangle_{\omega_0} dt\quad\forall x,y\in\Omega.
        \label{eq:loop space S1 r2n symplectic pairing}
    \end{equation}
    \Cref{eq:loop space S1 r2n symplectic pairing} can be rewritten explicitly as the sum of half-determinants, which we now give
    \begin{equation}
        A(x,y) = 2^{-1}\int_{S^1} \sum_{i=\underline{n}}\det\qty(\mqty[\mathring{x}_i & y_i \\ \mathring{x}_{n+i} & y_{n+i}]).
        \label{eq:symplectic pairing loops r2n determinant formula}
    \end{equation}
    A simple application of Holder's inequality will show that, for every $x,y\in\Omega$, we have
    \[
        \abs{A(x,y)}\leq 2^{-1}\sum_{i=\underline{n}}\norm{y_i\mathring{x}_{n+i}}_{L^2} + \norm{y_{n+i}\mathring{x}_{i}}_{L^2}.
    \]    
\end{definition}
\begin{definition}[Symplectic Action]\label{def:symplectic action closed curves}
    The \emph{symplectic action} (on closed curves) is a mapping $A:\Omega\to\real$ which \textbf{computes the area swept by the curve}. For an arbitrary loop $\gamma\in\Omega$, its action $A(\gamma)$ is given by:
    \begin{equation}
        A(\gamma) = 2^{-1}\int_{S^1}\langle\mathring{\gamma},\gamma\rangle_{\omega_0} = 2^{-1}\int_{S^1}\sum_{i=\underline{n}}\det\qty(\mqty[\mathring{\gamma}^{i} & \gamma^{i}\\ \mathring{\gamma}^{n+i} & \gamma^{n+i}])dt.
        \label{eq:loop space S1 r2n symplectic action}
    \end{equation}
    Alternatively, let $\lambda$ be the $1$-form on $\realtn$ such that $\omega_0 = d\lambda$, then $A(\gamma) = \int_{\gamma}\lambda$ --- the proof for this is below.
\end{definition}
\begin{note}[Symplectic Action in terms of $\lambda$]
    An easy computation in coordinates will show
    \begin{equation}
        A(\gamma) = 2^{-1}\int_{S^1} \sum_{i =\underline{n}} \gamma^i\mathring{\gamma}^{n+i} - \mathring{\gamma}^{i}\gamma^{n+i}.
        \label{eq:action lambda step1}
    \end{equation}
    Notice that the first term $\int_{S^1} \sum_{i=\underline{n}} \gamma^i\mathring{\gamma}^{n+i}$ is equal to $2^{-1}\int_\gamma \lambda$. Indeed,
    \[
        \int_{\gamma}\lambda = \int_{0}^1 \lambda(\gamma(t))(\mathring{\gamma}(t)) dt = \int_{S_1} \sum_{i=\underline{n}}\gamma^i\mathring{\gamma}^{n+i}.
    \]
    Using integration by parts, the integral over the each of the second terms in \cref{eq:action lambda step1} evaluates to
    \[
        2^{-1}\int_{S^1} \sum_{i=\underline{n}} \mathring{\gamma}^{i}\gamma^{n+i}=2^{-1}\sum_{i=\underline{n}}\gamma^{i}\gamma^{n+i}\eval_{\partial S^1} - 2^{-1}\int_{S^1}\sum_{i=\underline{n}}\mathring{\gamma}^{n+i}\gamma^{i}.
    \]
    The boundary terms disappear since $\gamma$ is periodic, and we notice that the left hand side of \cref{eq:action lambda step1} is the sum of $2^{-1}\int_\gamma\lambda + 2^{-1}\int_{\gamma}\lambda$, and the proof is complete.
\end{note}
%
%
%
\begin{remark}[General loops with period $L$]
    More generally, if we have two loops of period $L$ \cref{eq:loop space S1 r2n symplectic action} suggests that we can descend $\omega_0$ to an even larger space.
    \begin{equation}
        \Omega_{[0,L]} = C^\infty(\real/L\mathbb{Z}, \realtn)
        \label{eq:[0,L] r2n space}
    \end{equation}
    with $A_{[0,L]}(\gamma,\eta) = A(\gamma(Lt),\eta(Lt))$ which evaluates to
    \begin{equation}
     A_{[0,L]}(\gamma,\eta) = \frac{1}{2L}\int_{0}^{L}\langle \mathring{\gamma}(t),\eta(t)\rangle_{\omega_0}dt
     \label{eq:[0,L] r2n symplectic action}
    \end{equation}
\end{remark}

%
%
%
%
%
%
\begin{note}[$L^2$ descent of bilinear forms]
    The argument in this section about descending symplectic (resp. orthogonal) geometries onto $L^2$ functions \emph{into} the space is one of the reasons why $L^2$ functions are of such importance. To recapitulate:
    \begin{itemize}
        \item Given a bilinear form $\upsilon$ on $\real$, we can extend it to $\realtn$ for $n\geq 1$ using a 'hyperbolic decomposition' similar to \cref{eq:std symplectic form r2n 1}.
        \item This bilinear form on $\realtn$ descends into a bilinear form on the space of $L^2$ $1$-periodic loops from $S^1$ into $\realtn$,
        \item Since every loop with period $L$ admits a $1$-periodic representation, $\upsilon$ further descends to a bilinear form on $L^2([0,L], \realtn)$. Whose action is defined by the integral
        \[
            \langle \gamma,\eta\rangle_{\upsilon} = \frac{1}{2L}\int_{0}^{L}\langle\gamma(t),\eta(t)\rangle_{\upsilon}dt
        \]
    \end{itemize}
\end{note}
\begin{wts}[WC Reduction 2 --- Independence of the Symplectic Structure]\label{thm:wc reduction 2 symplectic structure }
    Suppose $(M,\omega)$ and $(N,\eta)$ are symplectic manifolds modelled on $\realtn$, and $u:M\to N$ is a symplectomorphism. 
    \begin{quote}
        To every function $F\in C^\infty(N)$, the vector field pullback of the Hamiltonian flow of $F$ is equal to the Hamiltonian flow of its pullback through $u$.     
    \end{quote}
    More precisely, if $X_F = \eta^{\wedge}(dF)$ and $u^*F = F\circ u$, we claim that
    \begin{equation}
        \omega^{\wedge}(d(u^* F)) = u^*(\eta^{\wedge}(dF))\qqtext{where} u^*(\eta^{\wedge}(dF)) = du^{-1}\circ X_F\circ u.
        \label{eq:reduction 2 eq1}
    \end{equation}
    If $\gamma$ is an integral curve of $X_{F\circ u}$, then $u\circ \gamma$ is an integral curve of $X_F$, and if $\varphi(s,x)$ and $\theta(t,y)$ denote the flows of $X_{F\circ u}$ and $X_{F}$, they relate to each other by $u$-conjugation as in \cref{eq:reduction 2 eq2}
    \begin{equation}
        u\circ\varphi^t = \theta^t\circ u.
        \label{eq:reduction 2 eq2}
    \end{equation}
\end{wts}
\begin{proof}
    Let $F$ be fixed, and write $A = dF$. Recall $d(F\circ u) = dF\circ du$. It suffices to show that
    \begin{equation}
        \omega^{\wedge}(u^* A) = u^*(\eta^{\wedge} A).
        \label{eq:reduction 2 eq3}
    \end{equation}
    We will show the left and right hand sides are equal at every tangent space. Given $p\in M$, we write
    \[X_p = \omega^{\wedge}(u^* A)(p) \qqtext{and} Y_p = u^*(\eta^{\wedge} A)(p).\]
    If $Z_p\in T_pM$ is arbitrary, we compute $\omega(p)(X_p, Z_p)$ and $\omega(p)(Y_p, Z_p)$ and the proof is complete upon showing equality. Now,
    \[\omega(p)\biggl(X_p, Z_p\biggr) = \eta(u(p))\biggl(du(p)[X_p,Z_p]\biggr)= A(u(p))\biggl(du(p)(Z_p)\biggr).\]
    Using the same technique of exchanging $\omega(p)$ with $\eta(u(p))$, we get
    \[
        \omega(p)\biggl(Y_p, Z_p\biggr) = \eta(u(p))\biggl(du(p)[Y_p, Z_p]\biggr),
    \]
    since $Y_p = (du^{-1}\circ \eta^{\wedge}A \circ u)(p)$, we obtain
    \[
        \omega(p)\biggl(Y_p, Z_p\biggr) = \eta(u(p))\biggl(\eta^{\wedge}A(u(p)),\: du(p)(Z_p)\biggr),
    \]
    which implies $X_p = Y_p$. This proves the first claim, and 
    \[
        du\circ X_{F\circ u} = X_F\circ du.
    \]
    Next, if $\gamma$ is an integral curve of $X_{F\circ u}$, then $\mathring{\gamma}(t) = X_{F\circ u}(\gamma(t))$ implies
    \begin{align*}
        \dv{t}u\circ \gamma(t)\eval_t &= du(\gamma(t))(X_{F\circ u}) = du\circ X_{F\circ u}\eval_{\gamma(t)} \\
        &= (X_F\circ u)\eval_{\gamma(t)} = X_F\eval_{u\circ \gamma(t)}
    \end{align*}
    and $u\circ \gamma$ is an integral curve of $X_F$. \Cref{eq:reduction 2 eq3} is proven upon realizing that $X_F$ and $X_{F\circ u}$ are $u$-related (Theorem 9.13 of \cite{Lee2013Introduction}).
\end{proof}
\begin{corollary}[WC Reduction 2.5 --- Periodic Orbits]\label{cor:wc reduction 2.5}
    Let $M,N$ and $u$ be as in \Cref{thm:wc reduction 2 symplectic structure }, the periodic orbits of $X_F$ correspond bijectively to the periodic orbits of $X_{F\circ u}$ through conjugation; and so do their periods.
\end{corollary}
\topheader{Quadratic Forms}
Let $n\geq 1$ be fixed, a \emph{quadratic form} on $\realn$ is a mapping $q:\realn\to\real$ where $q(rx) = \abs{r}^2q(x)$ for every $r\in\real$ and $x\in \realn$ and
\[
   \langle x,y\rangle_{q} = q(x+y) - [q(x)+q(y)]\quad\text{is a symmetric bilinear form.}
\]
% I have no clue why this is the definition.
If $q$ is a quadratic form, it is \emph{positive definite} whenever $\langle \cdot,\cdot\rangle_q$ is. We denote the set of positive definite quadratic forms by $\mathbb{P}$. The following proposition shows that every $q\in\mathbb{P}$  can be symplectically diagonalized on $\realtn$. 
\begin{wts}[Symplectic diagonalization of positive definite quadratic forms]
    If $q\in \mathbb{P}$ , there exists a linear mapping $\varphi\in\operatorname{Sp}(n)$ such that $q\circ \varphi$  takes on the form:
    \begin{equation}
    q\circ\varphi(x) = \sum_{i=\underline{n}} \frac{x_i^2 + x_{n+i}^2}{r_i^2}\qqtext{where} 0<r_1\leq r_2\leq\cdots\leq r_n.
    \label{eq:quadratic form symplectically diagonalized 1}
\end{equation}
    We call \cref{eq:quadratic form symplectically diagonalized 1} the \emph{normal form} of $q$.
\end{wts}
\begin{proof}
    Postponed for now.
\end{proof}
\begin{definition}[Associated open ellipsoid]
    Let $q$ be a positive definite quadratic form on $\realtn$; its \emph{asscoiated open ellipsoid}  is the subset $\Epsilon_q=[q<1]$. If $q$ is given in normal coordinates,
    \[
        \Epsilon_q = \bigset{x\in\realtn,\:\sum_{i=\underline{n}} r_j^{-2}(x_j^2 + x_{n+j}^2)<1}, \qqtext{and} \partial \Epsilon_q = \{x\in\realtn, \: q(x) = 1\}.
    \]
\end{definition}
\topheader{Orbits on Ellipsoids}
We see that reparameterization by a diffeomorphism $\alpha$ does not affect the \textbf{magnitude} of the symplectic action. Indeed, if $\gamma$ is a closed characteristic, and $\alpha$ a reparameterization, then 
\[
  A(\gamma)=\int_{\gamma}\lambda = \pm \int_{\gamma\circ\alpha}\lambda = \pm A(\gamma\circ \alpha).
\]
In the previous chapter, we have also proven that symplectomorphisms on $\realtn$ leave the action invariant; so it makes sense to speak of the action across closed characteristics on an energy surface (on $\realtn$). Our main result in this section concerns the \textbf{periodic orbits} of Hamiltonian flows on the boundaries of ellipsoids. \\

We will prove \Cref{thm:symplectic action on the boundary of ellipsoids} in a few steps.
\begin{wts}[Action on the boundary of ellipsoids]\label{thm:symplectic action on the boundary of ellipsoids}
    Let $q$ be a positive definite quadratic form, (not necessarily normal with respect to standard coordinates), then:
    \[
        \pi r_1^2 = \inf\bigset{\abs{A(\gamma)},\: \parbox{15em}{$\gamma$ is a non-degenerate closed characteristic of $\partial \Epsilon_q$.}},
    \]
    and the infimum is attained.
\end{wts}
Before proceeding any further, we will need to work out some computations involving symplectic bases (see \cite{Roman2007Advanced}). Let $x = (x_{\underline{n}}, x_{n+\underline{n}})\in\realtn$, we write 
\[
    \sum_{s} (x, \cl{x}) = \sum_{s}\begin{bmatrix}
        x_i \\ x_{n+i}
    \end{bmatrix}\qqtext{to mean} \sum_{i=\underline{n}}\sybasis\begin{bmatrix}
        x_i \\ x_{n+i}
    \end{bmatrix}.
\]
The symbols $x$ and $\cl{x}$ refer to the entries $x_i$ and $x_{n+i}$ within the summation. The standard symplectic form $J_{2n} = J_2\otimes \id{\realn}$ acts on $x$ in a convenient manner which justifies the decomposition:
    \[
        J_{2n}x = \sum_{i=\underline{n}}\sybasis\symatrix \begin{bmatrix}
            x_i \\ x_{n+i}
        \end{bmatrix} = \sum_{i=\underline{n}}\sybasis\begin{bmatrix}
            x_{n+i} \\ -x_{i}
        \end{bmatrix}.
    \]
    This reads, $\sum_s J_2 (x,\cl{x})=\sum_s (\cl{x},-x)$. Now, suppose we are given a Hamiltonian $F\in C^\infty(\realtn,\real)$, its Hamiltonian flow given by
    \[
        X_F = J\nabla F = \sum_{s}J_2 (\partial_{i}F, \partial_{n+i}F)
    \]

\begin{step}[Solutions of $X_q$ on $\partial \Epsilon_q$]
    Suppose $q = \sum_{i=\underline{n}}(x^2_{i} + x^2_{n+i})$ is in normal form. Every solution $\gamma$ on $\partial \Epsilon_q$ is given by
    \[
        \gamma(t) = \sum_{s} \begin{bmatrix}
            c_{\lambda_i}(t) & s_{\lambda_i}(t) \\
            -s_{\lambda_i}(t) & c_{\lambda_i}(t) 
        \end{bmatrix}\begin{bmatrix}
            \gamma_i(0) \\\gamma_{n+i}(0)
        \end{bmatrix},
    \]
    or equivalently: 
    \[
    \gamma(t) = \sum_{s}\exp(\lambda_iJ_2 t)(\gamma_i(0),\gamma_{n+i}(0)).
    \]
\end{step}
\begin{proof}[Proof of Step 1]
    Making the substitution, $\lambda_i = 2{r_i}^{-2}$, we can rewrite \cref{eq:quadratic form symplectically diagonalized 1}, which is convenient when we compute the gradient of $q$.
\begin{equation}
    q(x) = 2^{-1}\sum_{i=\underline{n}}\lambda_i(x_i^2 + x_{n+i}^2)\qqtext{where} 0<\lambda_n\leq\lambda_{n-1}\leq\cdots\leq\lambda_1.
    \label{eq:quadratic form symplectically diagonalized 2}
\end{equation}
Suppose $q$ is given by the right hand side of \cref{eq:quadratic form symplectically diagonalized 2}, then $\nabla q(x) = \diag(\lambda_{\underline{n}},\lambda_{\underline{n}})x$; or
\[
    \nabla q(x) = \sum \lambda_i(x_ie_i + x_{n+i}e_{n+i}) \qqtext{and}X_q(x) = \sum\lambda_i(x_{n+i}e_i - x_ie_{n+i}).
\]
If $\gamma(t) = (\gamma_1,\ldots,\gamma_{2n})$ is a solution to $X_q$ it must satisfy $\mathring{\gamma} = X_q(\gamma)$. It follows that
\[
\sum \mathring{\gamma}_ie_{i} + \mathring{\gamma}_{n+i}e_{n+i} =\sum_s\lambda_iJ_2(\gamma_i(0),\gamma_{n+i}(0)).
\] 
Comparing coefficients, we see that for $j = \underline{n}$, and $z_j(t) = (\gamma_j, \gamma_{n+j})\in\real^2$:
\[
    \mathring{z}_j = \lambda_j J_2 z_j\qqtext{implies} z_j(t) = \exp(\lambda_j J_2 t)z_j(0).
\]
Each $z_j$ has eigenvalues $\pm i\lambda_j$, where $i = \sqrt{-1}$ in this context. Computing $\exp(\lambda_j J_2 t)$ with the eigenvalues gives us
\[
    \exp(\lambda_j J_2 t) = \begin{bmatrix}
        \cos(\lambda_j t) & \sin(\lambda_j t)\\
        -\sin(\lambda_j t) & \cos(\lambda_j t)
    \end{bmatrix},\quad\text{which has period }2\pi \lambda_j^{-1} = \pi r_j^2.
\]
\begin{note}[Matrix exponentials]
    Write $J_2 = \symatrix$, and $J = J_{2n}$ is still the standard symplectic form on $\realtn$, then
    \begin{equation}
        \exp(\lambda J_2 t) = \begin{bmatrix}
            \cos(\lambda t) & \sin(\lambda t)\\
            -\sin(\lambda t) & \cos(\lambda t)
        \end{bmatrix} = \cos(\lambda t)\id{\real^2} + \sin(\lambda t)J_2;
        \label{eq:matrix exponential j2 lambda}
    \end{equation}
    and for the general case with the substitution $c_\lambda(t) = \cos(\lambda t)$, (resp. $s_\lambda(t)$):
    \begin{equation}
        \exp(\lambda J_{2n} t) = \begin{bmatrix}
            c_\lambda \id{\realn} & s_\lambda \id{\realn} \\ 
            -s_\lambda \id{\realn} & c_{\lambda} \id{\realn}
        \end{bmatrix} = \cos(\lambda t)\id{\realtn} +\sin(\lambda t)J_{2n}.
        \label{eq:matrix exponential j r2n lambda}
    \end{equation}
    We offer a quick proof. The alternate symplectic form $\wig{J}_{2n} = \id{\realn}\otimes J_2$ is block diagonal, with eigenvalues $\pm \lambda_i \sqrt{-1}$. It follows that $\exp(\lambda\wig{J}_{2n}t) = \diag_{i=\underline{n}}(\exp(\lambda_i J_2 t))$. It is clear that $\wig{J}_{2n}$ and $J_{2n}$ differ by a relabelling of the basis vectors. More precisely:
    \[
        \wig{J}x = \sum_{i=\underline{n}} \begin{bmatrix}
            e_{2i-1} & e_{2i}
        \end{bmatrix}\symatrix\begin{bmatrix}
            x_{2i-1} \\ x_{2i}
        \end{bmatrix},
    \]
    which differs by the basis change $(e_i, e_{n+i})\mapsto (e_{2i-1}, e_{2i})$.
\end{note}
% \begin{remark}
%         Is the following true? For every $A\in\real^{n\times n}$, we have 
%         \[
%             \exp(At)\otimes \id{\realn} = \exp((A\otimes \id{\realn})t)
%         \]
%         Possible hint: Differentiate both sides under a test vector $x_0\in\realn$.\\

%         An interpretation I particularly like is that by differentiating the matrix exponential, one differentiates in the sense of distributions.
%     \end{remark}
Suppose $\gamma(0) = (\gamma_{\underline{2n}}(0))\in\partial\Epsilon_q$, then the integral curve generated by $\gamma(0)$ is the sum of the eigenmodes $\pm\lambda_i$ and this proves the claim claim. Finally, we note that because $0<r_1\leq r_2\leq\cdots\leq r_n$, the minimum period of an orbit is $\pi r_1^2$. 
\end{proof}
\begin{step}[Action of periodic orbits of $X_q$ on $\partial \Epsilon_q$]
    Let $q(x) = \sum_{i=\underline{n}} r_i^{-2}(x^2_i + x^2_{n+i})$ be a positive definite ellipsoid in normal form on $(\realtn,\omega_0)$, then
    \[
        \pi r_1^2 = \inf\bigset{\abs{A(\gamma)},\: \parbox{12em}{$\gamma$ is a periodic orbit of $X_F$ on $\partial \Epsilon_q$.}}.
    \]
\end{step}
\begin{proof}[Proof of Step 2]
    Let $\gamma(t)$ be such a curve on $\partial \Epsilon_q$, and define $z_i = (\gamma_i(0),\gamma_{n+i}(0))$ for $i = \underline{n}$. Notice that $\dv{t}e^{\lambda_i J_2 t} = \lambda_i J_2e^{\lambda_i J_2 t}$, and from Step 1, we compute the integrand of $A(\gamma)$:
    \begin{align*}
        2^{-1}\langle \mathring{\gamma}(t),\gamma(t)\rangle_{\omega_0} &= 2^{-1}\sum \left\langle \lambda_i J_2 e^{\lambda_i J_2 t} {z}_i,\quad J_2e^{\lambda_i J_2 t}{z}_i\right\rangle_{\realtn}\\[2ex]
        &= 2^{-1}\sum \lambda_i\abs{{z}_i}^2=q(\gamma(t))=1
    \end{align*}
    Hence, $A(\gamma) = L$ where $L$ is the period of $\gamma$, and is minimized whenever $L = \pi r_1^2$. For an arbitrary $q$, there exists a linear symplectic mapping $\varphi$ such that $\varphi\circ q$ is in normal form. Since $\varphi$ preserves the symplectic action, this proves Step 2.
\end{proof}
\begin{step}[Action of arbitrary periodic orbits on $\partial \Epsilon_q$]
    Let $q\in\mathbb{P}$, and $F$ define the boundary of $\Epsilon_q$, then
    \[
        \pi r_1^2 = \inf\bigset{\abs{A(\gamma)},\: \parbox{12em}{$\gamma$ is a periodic orbit of $X_F$ on $\partial \Epsilon_q$.}}.
    \]   
\end{step}
\begin{proof}[Proof of Step 3]
    From \Cref{thm:wc reduction 1 hamiltonian}, we know that the orbits of $X_q$ and $X_F$ on $\partial \Epsilon_q$ correspond one-to-one with each other. Given an orbit, $\gamma$ of  $X_F$, the composition $\gamma\circ\alpha$ is an orbit of $X_q$ for some diffeomorphism $\alpha$. Therefore, $A(\gamma) = A(\gamma\circ\alpha) = \pm A(\gamma)$ and the proof is complete.
\end{proof}
\clearpage

%
%
\fchapter{7: Symplectic Capacities}
\topheader{Introduction}
%% Maybe add that the total volume of ?????
We will discuss a class of correspondences from the set of all symplectic manifolds to $[0,+\infty]$ --- similar to the total measure or volume of a space --- but are preserved under symplectomorphisms. 
\begin{wts}[Open submanifolds are symplectically embedded]\label{thm:open submanifolds symplectically embedded}
    If $U$ is an open subset of a symplectic manifold $(M,\omega)$, then $(U,\omega)$ is again a symplectic submanifold, and the inclusion map $\iota_U$ is a symplectic embedding.
\end{wts}
\begin{proof}
    It is clear that $U$ is an embedded submanifold of $M$ by elementary manifold theory. Because $U$ has codimension $0$, $d\iota_U\cong\id{\realtn}$ is the identity between abstract tangent spaces. Therefore the pullback $\iota_U^*(\omega) = \omega$ as needed. 
\end{proof}
\begin{lemma}[Symplectic isomorphisms and dilations]\label{lem:symp. isomorphisms and dilations}
    Let $(U,\omega_0)$ be an open symplectic submanifold of $(\realtn,\omega_0)$. For every $\alpha\neq 0$, $(\alpha U,\omega_0)$ is symplectically isomorphic to $(U,\sgn(\alpha)\abs{\alpha}^2\omega_0)$, where $\alpha U=\{\alpha x,\: x\in U\}=\{x,\: \alpha^{-1}x\in U\}$.
\end{lemma}
\begin{proof}
    We construct a mapping which relates $\alpha U$ with $U$. Define $\varphi:\realtn\to\realtn$ where $\varphi(x) = \alpha^{-1}x$. Its Jacobian is simply $\alpha^{-1}\id{\realtn}$ at every point, and it is clear that $\varphi\vert_{\alpha U}$ is a diffeomorphism onto $U$. Next, we claim that $(\alpha U,\omega_0)$ is symplectically isomorphic to $(U,\sgn(\alpha)\abs{\alpha}^2\omega_0)$. This is easy to see, because the differential of $\varphi$ is the linear map $\id{\realtn}{\alpha^{-1}}$, and the two $\alpha$s pop out by bilinearity within the tensor pullback. Indeed, fix any $\alpha x\in \alpha U$, then 
    \[
        \varphi^*(\sgn(\alpha)\abs{\alpha}^2\omega_0)(\alpha x)(v_1,v_2) = (\sgn(\alpha)\abs{\alpha}^2\omega_0)(x)\biggl(\alpha^{-1}v_1, \alpha^{-1}v_2\biggr) = \omega_0(x)(v_1,v_2).
    \]
\end{proof}
We define two very special symplectic manifolds, the \emph{open $r$-ball} and the \emph{open $r$-cylinder}:
\[
    B(r) = \bigset{x\in\realtn,\: \sum_{i=\underline{n}} x^2_{i} + x^2_{n+i}=\abs{x}^2<r^2}\qqtext{and}Z(r) = \bigset{x\in\realtn,\: x^2_1 + x^2_{n+1}<r^2};
\]
which are both equipped with the standard symplectic form $\omega_0$. We see that if $0<r_0\leq r_1$, $B(r_0)$ embeds symplectically into $B(r_1)$ (resp. $Z$); and it is also true that $B(r)$ is embedded in $Z(r)$. \Cref{lem:symp. isomorphisms and dilations} also gives us the diagram in \cref{fig:dilation-balls-diagram}.

% https://q.uiver.app/#q=WzAsNCxbMSwxLCIoQihyKSwgXFxzZ24oXFxhbHBoYSkoXFxhYnN7XFxhbHBoYX1eezEvMn0pXnsyfVxcb21lZ2FfMCkiXSxbMSwwLCIoXFxzZ24oXFxhbHBoYSlcXGFic3tcXGFscGhhfV57MS8yfUIociksIFxcb21lZ2FfMCkiXSxbMCwwLCIoXFxhYnN7XFxhbHBoYX1eezEvMn1CKHIpLCBcXG9tZWdhXzApIl0sWzAsMSwiKEIociksIFxcc2duKFxcYWxwaGEpKFxcYWJze1xcYWxwaGF9XnsxLzJ9KV57Mn1cXG9tZWdhXzApIl0sWzIsMSwiIiwwLHsic3R5bGUiOnsidGFpbCI6eyJuYW1lIjoiYXJyb3doZWFkIn19fV0sWzEsMCwiIiwwLHsic3R5bGUiOnsidGFpbCI6eyJuYW1lIjoiYXJyb3doZWFkIn19fV0sWzAsMywiIiwwLHsic3R5bGUiOnsidGFpbCI6eyJuYW1lIjoiYXJyb3doZWFkIn19fV0sWzIsMywiIiwyLHsic3R5bGUiOnsidGFpbCI6eyJuYW1lIjoiYXJyb3doZWFkIn0sImJvZHkiOnsibmFtZSI6ImRvdHRlZCJ9fX1dXQ==
\begin{figure}[!h]
\centering
\begin{tikzcd}
	{(\abs{\alpha}^{1/2}B(r), \omega_0)} & {(\sgn(\alpha)\abs{\alpha}^{1/2}B(r), \omega_0)} \\
	{(B(r), \sgn(\alpha)(\alpha\omega_0)} & {(B(r), \sgn(\alpha)(\abs{\alpha}^{1/2})^{2}\omega_0)}
	\arrow[tail reversed, from=1-1, to=1-2]
	\arrow[tail reversed, from=1-2, to=2-2]
	\arrow[tail reversed, from=2-2, to=2-1]
	\arrow[dotted, tail reversed, from=1-1, to=2-1]
\end{tikzcd}
\caption{Dilations of $B(r)$, where symplectic embeddings are represented by arrows.}
\label{fig:dilation-balls-diagram}
\end{figure}

\FloatBarrier



\begin{wts}[Darboux's Theorem]\label{wts:darbouxs theorem}
    Let $(M,\omega)$ be a symplectic manifold modelled on $\realtn$. At every point $p\in M$, there exists a chart $\varphi: U\to\hat{U}$ where its \textbf{inverse} satisfies
    \[
        (\varphi^{-1})^*\omega = \omega_0.
    \]
\end{wts}
\begin{proof}
    Postponed.
\end{proof}
\topheader{Definition of a capacity}
\begin{definition}[Symplectic capacity]
    A \emph{symplectic capacity} $\frakc$ is a function that assigns to each symplectic manifold $(M,\omega)$:  a number $\frakc(M,\omega)\in[0,+\infty]$ satisfying the following properties
    \begin{enumerate}
        \item Monotonicity: Given two symplectic manifolds $(M,\omega)$ and $(N,\eta)$ \textbf{of the same dimension}, if $(M,\omega)$ embeds symplectically into $(N,\eta)$, then $\frakc(M,\omega) \leq \frakc(N,\eta)$.
        \item Conformality: If $\alpha\neq 0$ is a real number, then $\frakc(M,\alpha\omega) = \abs{\alpha}\frakc(M,\omega)$.
        \item Non-triviality: The capacities of $B(1)$ and $Z(1)$ are equal to $\pi$, \textbf{across all $n$}.
    \end{enumerate}
\end{definition}
It is clear that, if two symplectic manifolds are symplectically isomorphic, then their symplectic capacities must agree. Furthermore,
\begin{wts}[Capacities of dilated subsets of $\realtn$]\label{thm:symplectic capacity scaling open subsets}
    Let $U$ be an open subset of $\realtn$, then $(U,\omega_0)$ is a symplectic manifold that is symplectically embedded into $(\realtn,\omega_0)$, and $\frakc(\alpha U,\omega_0) = \abs{\alpha}^2\frakc(U,\omega_0)$ for all $\alpha\neq 0$, where $\alpha U = \{x,\: \alpha^{-1} x\in U\}$.
\end{wts}
\begin{proof}
    First, every $U\osub\realtn$ is an embedded submanifold of $\realtn$, with the inclusion $\iota_{U} = \id{U}$. At every point $x\in U$: we see that $\iota_U^{*}(\omega_0)(x)(\cdot,\cdot) = \omega_0(x)(d\id{U}(\cdot,\cdot))$ which is equal to the symplectic form on $U$. Given a capacity $\frakc$, $(\alpha U,\omega_0)$ is again an open submanifold of $\realtn$, for all $\alpha\neq 0$; so $\frakc(\alpha U,\omega_0)$ makes sense. By conformality of $\frakc$, we see that $\frakc(\alpha U,\omega_0) = \frakc(U,\pm\abs{\alpha}^2\omega_0) = \abs{\alpha}^2\frakc(U,\omega_0)$.
\end{proof}
The following Proposition uses the fact that the non-triviality and the conformality properties of $\frakc$ means that the capacity of any open subset $U$, that is squeezed between 
\[
    B(r)\subseteq U\subseteq Z(r),\quad \text{is precisely }\pi r^2.
\]
Together with the symplectic invariance of capacities, we have:
\begin{wts}[Capacities of ellipsoids of $\realtn$]
    Let $\frakc$ be a capacity, then $\frakc(\Epsilon,\omega_0) = \pi r_1^2$ for every open ellipsoid $\Epsilon$ with $r(\Epsilon) = (r_1,\ldots, r_n)$.
\end{wts}
\begin{proof}
    By \Cref{thm:symplectic capacity scaling open subsets}, we see that
    \begin{align*}
        \frakc(B(r),\omega_0) &= \abs{r}^2\frakc(B(1),\omega_0) = \pi\abs{r}^2\\
        \frakc(Z(r),\omega_0) &= \abs{r}^2\frakc(Z(1),\omega_0) = \pi\abs{r}^2.
    \end{align*}
    There exists a linear symplectic isomorphism that puts $(\Epsilon,\omega_0)$ in normal form --- with $\varphi(\Epsilon,\omega_0) = (\Epsilon_{\mathrm{normal}}, \omega_0)$, and because $U\osub V\osub \realtn$ means $U$ symplectically embeds into $V$, and
    \[
        \frakc(\varphi B(r_1),\omega_0)\leq \frakc(\varphi\Epsilon,\omega_0)\leq \frakc(\varphi Z(r_1),\omega_0)\qqtext{implies} \frakc(\Epsilon,\omega_0) = \pi r_1^2.
    \]
\end{proof}
\begin{remark}[Capacities of bounded subsets of $\realtn$]
    By monotonicity, one sees that if $U$ is an open, precompact subset of $\realtn$, $0<\frakc(U,\omega_0)<+\infty$ for every capacity $\frakc$.. However, there are compact symplectic manifolds which have infinite capacity.
\end{remark}
To sum up the first two sections, we have explored the different ways symplectic manifolds are embedded into each other. Open subsets of symplectic manifolds play a key role in this, and in the case of $\realtn$: we can dilate open subsets by $\alpha$ at the cost of a factor of $\sgn{\alpha}\abs{\alpha}^2$ on $\omega_0$. One should interpret this as $\omega_0$ being some kind of area. We have also defined what it means for a function to be a symplectic capacity, and by specifying its values on $B(1)$, and $Z(1)$, we specify its values on all ellipsoids and 'ellipsoid-like' open subsets of $\realtn$.\\

If we assume the existence of a capacity, we obtain an infamous result in symplectic topology.
\begin{wts}[Gromov's Squeezing Theorem]\label{thm:Gromov's Squeezing Theorem}
     Assuming the existence of a capacity $\frakc$, given positive numbers $r_0$, $r_1$, the open ball $B(r_0)$ embeds into $Z(r_1)$ symplectically if and only if $r_0\leq r_1$.
\end{wts}
\begin{proof}[Proof assuming $\frakc$]
    The 'if' direction follows from \Cref{thm:open submanifolds symplectically embedded}. Conversely, suppose $B(r_0)\hookrightarrow Z(r_1)$, then $\frakc(B(r_0),\omega_0)\leq \frakc(Z(r_1),\omega_0)$ by monotonicity.
\end{proof}
%
%
%
\topheader{Gromov's Width}
We give an example of a symplectic capacity, whose proof depends on \Cref{thm:Gromov's Squeezing Theorem}. First, we need a small, but useful definition.
\begin{definition}[Increasing/decreasing correspondence]
    If $A$ and $B$ are non-empty subsets of $\real$, an \emph{increasing correspondence} from $A$ to $B$ (resp. \emph{decreasing}) is a mapping $f: A\to B$ where $\id{A}\leq f$ (resp. $f\leq \id{A}$).
\end{definition}
It follows that if there exists an increasing correspondence from $A$ to $B$, then $\sup A\leq \sup B$; and if $A\subseteq B$, then $\sup A\leq \sup B$, this follows from the previous claim with $f =\id{A}$.
\begin{definition}[Gromov's Width]
    If $(M,\omega)$ is a symplectic manifold modelled on $\realtn$, its \emph{Gromov's width} is the number 
    \[
        \Grom(M,\omega) = \sup\bigset{\pi r^2,\:(B(r),\omega_0)\hookrightarrow (M,\omega) \text{ symplectically}.}.
    \]
\end{definition}
\begin{wts}[Properties of Gromov's Width]\label{thm:properties of gromov width}
    Gromov's width is a symplectic capacity, and it is minimal: $\Grom(M,\omega) \leq \frakc(M,\omega)$ for every symplectic manifold $(M,\omega)$ and capacity $\frakc$.
\end{wts}
\begin{proof}
    By Darboux's Theorem, if $(M,\omega)$ is a symplectic manifold, then $\{\pi r^2, \: B(r)\hookrightarrow (M,\omega) \text{ symplectically.}\}$ is non-empty, and $\Grom(M,\omega)$ is strictly positive.\\
    
    Let $(M, \omega)$, and $(N,\eta)$ be symplectic manifolds of dimension $2n$, where $\varphi: M\hookrightarrow N$ is a symplectic embedding. Given any open $r$-Ball $B(r)\hookrightarrow M$, it is immediate that $B(r)\hookrightarrow N$, as the composition of two composable symplectic embeddings is again a symplectic embedding. This induces an increasing correspondence, and proves monotonicity.\\

    Let $\alpha\neq 0$ be fixed, if $(B(r),\omega_0)\hookrightarrow (M,\omega)$, it is clear that we can dilate the symplectic forms on both sides, and obtain $(B(r),\alpha\omega_0)\hookrightarrow (M,\alpha\omega)$. However, \cref{fig:dilation-balls-diagram} shows that $(\abs{\alpha}^{1/2}B(r),\omega_0)\hookrightarrow(M,\alpha\omega)$. This implies $\abs{\alpha}\{\pi r^2, \: B(r)\hookrightarrow (M,\omega)\}$ is contained in $\{\pi r^2,\: B(r)\hookrightarrow (M,\alpha\omega)\}$ as a subset. Applying the monotonicity of the supremum gives us one direction of the estimate; and reversing the roles of the two manifolds establishes conformality.\\

    To show non-triviality, one sees that $B(1)$ embeds symplectically into itself, and $\pi=\Grom(B(1))\leq\Grom(Z(1))$. On the other hand, Gromov's Squeezing Theorem gives us $\Grom(Z(1))\leq \pi$.\\

    Let $\frakc$ be a symplectic capacity. If $B(r)\hookrightarrow M$, by non-triviality of $\frakc$ and by \Cref{thm:symplectic capacity scaling open subsets}, we see that $\pi r^2=\frakc(B(r),\omega_0)\leq \frakc(M,\omega)$. Since $\Grom(M,\omega)$ is the supremum over $\pi r^2$, we are done.
\end{proof}
Every capacity function $\frakc$ induces a smaller capacity which takes the monotonicity of the open embeddings into account. 
\begin{definition}[Inner capacity]
    If $\frakc$ is a capacity, the \emph{inner capacity} of $\frakc$ is a function that assigns to every symplectic manifold $(M,\omega)$ the number
    \[
        \frakc^{\vee}(M,\omega) = \sup\bigset{(U,\omega),\: U\osub M\text{ and hides in } M.}.
    \]
\end{definition}
\begin{definition}[Inner regularity of symplectic capacities]
    A symplectic capacity $\frakc$ is \emph{inner regular} whenever $\frakc^\vee = \frakc$.
\end{definition}
We note in passing that $\Grom$ is inner regular.
\begin{wts}[Properties of the inner capacity]
    The inner capacity is a symplectic capacity, and $\frakc^\vee\leq \frakc$ for every capacity $\frakc$.
\end{wts}
\begin{proof}
    Fix a capacity $\frakc$, monotonicity of $\frakc^{\vee}$ follows from that of the supremum, the same for conformality. Gromov's Squeezing Theorem tells us that $\frakc^{\vee}$ is non-trivial. Finally, $\frakc^{\vee}\leq\frakc$ is trivial.
\end{proof}
% Give example of a symplectic manifold with boundary.
% What about the set \{\abs{x}\leq 1\}?
\topheader{The Orbital Capacity}
Let $(M,\omega)$ be a symplectic manifold (possibly with boundary), we define a subspace of $C^\infty(M,\real)$ that will help us to view periodic orbits in a different angle, by leveraging a distinguished symplectic capacity. 
\begin{definition}[Regular Hamiltonian]\label{def:regular hamiltonians}
    A smooth function $H\in C^\infty(M,\real)$ is called a \emph{regular Hamiltonian}, if all of the following hold.
    \begin{enumerate}
        \item There exists an open subset $U\osub M$ where $H$ vanishes; or $H(U) = 0=\min(H)$.
        \item There exists a compact $K\subseteq M\setminus \partial M$, outside of which $H$ attains its maximum; or $H(M\setminus K) = \max(H)$.
    \end{enumerate}
    The set of all regular Hamiltonians of $(M,\omega)$ is hereinafter denoted by $\hcal(M,\omega)$; and the quantity $\osc(H) = \max(H) - \min(H)$ is called the \emph{$\frakc_0$-oscillation of $H$}.
\end{definition}
We point out that the notion of a regular Hamiltonian is a topological one. The Hamiltonian flow of an arbitrary $H\in\hcal(M)$ must be compactly supported, as $H$ is constant outside of $K$; and we sometimes write $\supp{X_H}$ to refer to the smallest compact set outside of which $H = \osc(H)$.
\begin{definition}[Admissible Hamiltonian]\label{def:admissible hamiltonians}
    A Hamiltonian $H\in \hcal(M,\omega)$ is \emph{admissible} if all periodic orbits of $X_H$ have period $T > 1$. The space of admissible Hamiltonians on $(M,\omega)$ is denoted by $\hcala(M,\omega)$.
\end{definition}
\begin{definition}[Orbital capacity]
    The \emph{orbital capacity} of a symplectic manifold (with or without boundary) $(M,\omega)$ is denoted by $\frakc_0(M,\omega)$ and is the quantity
    \[
        \frakc_0(M,\omega) = \sup\bigset{\osc(H),\: \parbox{17em}{$H$ is an admissible Hamiltonian of $M$.}}.
    \]
\end{definition}
\textbf{Our goal is to prove that $\frakco$ is an inner regular capacity}. The proof is extremely long, and will be divided into several chapters. 
\begin{wts}[Monotonicity of $\frakco$]\label{thm:monotonicity of frakco}
    If $\varphi: M\hookrightarrow N$ is a symplectic embedding between manifolds of dimension $2n$, there exists an increasing correspondence from $\{\osc(H),\: H\in\hcal_a(M)\}$ into $\{\osc(H),\: H\in\hcal_a(N)\}$; and $\frakco(M)\leq \frakco(N)$.
\end{wts}
\begin{proof}
We construct a correspondence from $\hcal(M)$ into $\hcal(N)$ --- keeping in mind that regularity is a purely topological phenomenon. For every regular Hamiltonian $H\in\hcal(M)$, we can extend $H$ to $\hcal(N)$ with
    \[
        H_{\varphi} = H\circ\varphi^{-1}\vert_{\varphi(M)} + \osc(H)\vert_{N\setminus \varphi(M)},\text{ with }\osc(H) = \osc(H_{\varphi}).
    \]
    The following note encapsulates this routine argument.
\begin{note}[Extension of regular Hamiltonians from submanifolds]
\begin{lemma}[Regularity of $H_{\varphi}$]
        Let $M$ and $N$ be symplectic manifolds of dimension $2n$, and $\varphi: M\hookrightarrow N$ a symplectic embedding. For every regular Hamiltonian $H\in\hcal(M)$, we can extend $H$ to $\hcal(N)$ with
        \[
        H_{\varphi} = H\circ\varphi^{-1}\vert_{\varphi(M)} + \osc(H)\vert_{N\setminus \varphi(M)},\text{ with }\osc(H) = \osc(H_{\varphi}).
        \]
\end{lemma}
\begin{proof}
    Let $K = \supp{X_H}$, because $M$ is LCH, we obtain a precompact $U\osub M\setminus\partial M$ with $K\subseteq U$. Every point in $U$ admits a $2n$-Euclidean neighbourhood, because $\varphi(M)$ is a submanifold (we assume all submanifolds are embedded); we obtain slice charts about every point $p\in \varphi(K)\subseteq\varphi(M)$ --- which is a submanifold of $N$ with codimension $0$. \\

    Because $\varphi(K)$ is compact in $N$, and its complement can be written as $N\setminus \varphi(M) + \varphi(M)\setminus\varphi(K)$; and a moment's thought will show that $H_{\varphi}\vert_{N\setminus \varphi(K)} = \osc(H)$. Next, we claim that $H_{\varphi}$ is a smooth extension. The manifold interior $M\setminus \partial M$ is an open submanifold of $N$ --- by appealing to a LCH argument --- we obtain an open, precompact subset $V$ where $K\subseteq V\subseteq \cl{V}\subseteq M\setminus\partial M$ that is open and precompact relative to $N$ as well.\\

    Let $H' = \osc(H) - H$ be defined on $K$, we can extend $H'$ to be a smooth function on $N$, where $H'\vert_{K} = \osc(H)- H$ and $\supp{H'}\subseteq V$. Taking the reflection of $\wig{H} = \osc(H) - H'$, we see that $\wig{H}\in C^\infty(N)$, and $\supp{X_{\wig{H}}}\subseteq \cl{V}$. Furthermore, because $\wig{H}\vert_{M\setminus\partial M} = H$, it follows $\supp{X_{\wig{H}}} = \supp{X_H}=K$. Notice if $W$ is an open subset on which $H$ vanishes, then $\wig{H}\vert_W = 0$ as well. Finally, $\wig{H} = H_{\varphi}$, this proves that $H_{\varphi}$ is regular in $N$ with $\osc(H) = \osc(H_{\varphi})$.
    \end{proof}
\end{note}
We now show that $H$ is admissible if and only if $H_{\varphi}$ is. The orbits of $H$ and $H_{\varphi}$ are contained in $M\setminus\partial M$. Since $\varphi\vert_{M\setminus\partial M}$ is a symplectic isomorphism, \Cref{cor:wc reduction 2.5} tells us that their orbits correspond to each other by conjugation of $\varphi$, and therefore their periods are identical.
\end{proof}
\begin{wts}[Conformality of $\frakco$]\label{thm:conformality of frakco}
    Let $(M,\omega)$ be a symplectic manifold. If $\alpha\neq 0$ and $F\in\hcal(M,\omega)$, we write $\wig{F}=\abs{\alpha}F$ --- which is in $\hcal(M,\omega)$, and $\osc{\wig{F}} = \abs{\alpha}\osc{F}$. If $\wig{X}_{\wig{F}}$ is the Hamiltonian flow of $\wig{F}$ under the dilated manifold $(M,\alpha\omega)$, then 
    \[
        \wig{X}_{\wig{F}} = (\sgn{\alpha})X_{F}.
    \]
    We also see that
    \begin{enumerate}
        \item $X_{F}$ and $\wig{X}_{\wig{F}}$ have identical closed (resp. closed and non-degenerate) characteristics. 
        \item The periodic orbits of $X_F$ and $\wig{X}_{\wig{F}}$ correspond to each other up to a change in orientation. That is, $\gamma(t)$ is a $T$-orbit of $X_F$ iff $\gamma(\sgn(\alpha)t)$ is a $T$-orbit of $\wig{X}_{\wig{F}}$; and hence 
        \item $X_F$ is admissible iff $\wig{X}_{\wig{F}}$ is.
    \end{enumerate}
    
\end{wts}
\begin{proof}
    We start with some important properties of dilations, beginning with a simple equation that follows from the $C^\infty(M)$-linearity of $\omega^{\wedge}$:
    \[
        X_{\alpha F} = \alpha X_{F}\quad\text{for every } \alpha\neq 0.
    \]
    The rest is given in the note below.
    \begin{note}[Dilations and Flows]
    \begin{lemma}[Dilation of Hamiltonian vs. Symplectic Structure]
        Let $(M,\omega)$ be a symplectic manifold and $\alpha\neq 0$. For any $F\in C^\infty(M)$, 
        \[
            \wig{X}_{F} = \alpha^{-1}X_{F},\quad\text{where } \wig{X}_{F} = (\alpha\omega)^{\wedge}(dF).
        \]
    \end{lemma}
    \begin{proof}
    If $(M,\omega)$ is a symplectic manifold, for any smooth function $F$ we have
    \[
    \omega(X_F, v) = dF(v)  = (\alpha\omega)(\alpha^{-1}X_{F}, v) = (\alpha\omega)(\wig{X}_F,v).
    \]
    \end{proof}
    \begin{lemma}[Dilation of Vector Fields]
        Let $M$ be an arbitrary manifold and $X\in \vField(M)$. If $\alpha\neq 0$, 
        \begin{itemize}
            \item a curve $\gamma(t)$ is a solution (resp. an orbit) to $X_F$ if and only if $\gamma(\alpha t)$ is a solution (resp. an orbit) to $\alpha X_F$; and
            \item $\gamma(t)$ is a $T$-orbit of $X_F$, if and only if $\gamma(\alpha t)$ is a $T\abs{\alpha}^{-1}$-orbit of $\alpha X_F$.
        \end{itemize}
    \end{lemma}
    \begin{proof}
        Mimicking the proof of \Cref{thm:wc reduction 1 hamiltonian}, the chain rule gives us the first claim; and the second follows from solving for $t$ within $\alpha t = (\sgn{\alpha})T$. 
    \end{proof}
\end{note}
Let $F$ be a regular Hamiltonian of $(M,\omega)$, it is clear that $\wig{F} = \abs{\alpha} F$ is in $\hcal(M,\alpha\omega)$ with oscillation as described in the statement of \Cref{thm:conformality of frakco}. It follows that $\wig{X}_{\wig{F}} = \alpha^{-1}X_{\wig{F}}= \sgn(\alpha) X_F$, and from which we can deduce the rest of the claims.
\end{proof}

\fchapter{8: The orbital capacity $\frakco$}
\topheader{Non-triviality Part 1}
This section's main result is to show
\[
    \pi \leq \frakco(B(1))\leq \frakco(Z(1))
\]
Will use a mollifier/convolution argument.
\topheader{Start of Non-Triviality Part 2}
wts
\[
    \frakco(Z(1))\leq \pi.
\]
\topheader{An engulfing ellipsoid}

\topheader{Critical points}
%
%
%%%

%
%
%
%
\fchapter{9: Applications of the orbital capacity $\frakc_0$}
\topheader{Introduction}
In this section, $(M,\omega)$ will always refer to a symplectic manifold.
\begin{remark}
    \begin{itemize}
        \item $\mathcal{I}$ refers to an open interval in $\real$, and
        \item $\mathcal{I}_0$ refers to an open interval in $\real$ containing the origin.
    \end{itemize}
\end{remark}
\begin{definition}[Parameterized family of hypersurfaces modelled on $S$]
    Let $S$ be a compact hypersurface of $(M,\omega)$, a parameterized family (of hypersurfaces) modelled on $S$ is a diffeomorphism $\Psi_S: S\times \mathcal{I}\to U\osub M$, where $U$ is an open neighbourhood of $S$ and $\mathcal{I}$ is an open interval containing the origin, and similar to a homotopy: $\Psi_S(\cdot,0) = \id{S}$.\\

    If $S$ is understood to be a hypersurface with a parameterized family, the notation $S_{\varepsilon}$ will always refer to  $\Psi_S(S\times\{\varepsilon\})$; and $S = S_0$.
\end{definition}
\begin{wts}[page 114]
    The following statements are equivalent:
    \begin{itemize}
        \item The line bundle $\mathcal{L}_S\to S$ is orientable
        \item The normal bundle $N_S\to S$ is orientable,
        \item $S$ is orientable,
        \item There exists a parameterized family of hypersurfaces modelled on $S$,
        \item There exists a smooth function $H\in C^\infty(U,\real)$ where $S\subseteq U\osub M$ such that $dH\vert_U\neq 0$ and $S = H^{-1}(c)$ for some constant $c\in\real$.
    \end{itemize}
\end{wts}   
\topheader{Hypersurfaces that are boundaries of symplectic manifolds}
In this section, we only consider hypersurfaces that are the boundary to a compact symplectic manifold. If $S$ is the manifold boundary of $(B,\omega)$, and $S$ has a parameterized family, then every $S_{\varepsilon}$ is the manifold boundary of $(B_\varepsilon,\omega)$, and we can assume (?) that these symplectic manifolds are nested as follows
\[
    B_{\varepsilon}\hookrightarrow B_{\varepsilon'}\quad\forall\varepsilon\leq\varepsilon'.
\]
And by monotonicity of $\frakc_0$: $\frakc_0(B_{\varepsilon},\omega)\leq\frakc_{0}(B_{\varepsilon'},\omega)$.
\begin{definition}[Orbital-Lipschitz hypersurfaces]
    A compact hypersurface $S_{\varepsilon^*}$ is of orbital-Lipschitz (or $\frakc_0$-Lipschitz) type, if there are positive constants $L, \mu$ where
    \[
        C(\varepsilon)\leq C(\varepsilon^*) + L(\varepsilon - \varepsilon^*)\quad\forall \varepsilon \in [\varepsilon^*,\: \varepsilon^* +\mu],
    \]
    where $\frakc_0(B_{\varepsilon},\omega) = C(\varepsilon)$.
\end{definition}
\begin{wts}[Theorem 3. page 116]
    Let $(M,\omega)$ be a symplectic manifold with finite orbital capacity. Given an energy surface $S\subseteq M$ that is 1) the boundary of a symplectic manifold, and 2) is of $\frakc_0$-Lipschitz type, then
    \[
        \Periodic(S)\neq \varnothing.
    \]
\end{wts}
\begin{remark}
    This means, there exists a small interval to the right of $\varepsilon^*$ where the symplectic capacities of the manifolds $B_{\varepsilon}$ are controlled linearly a linear term:
    \[
        \abs{C(\varepsilon) - C(\varepsilon^*)}\leq L(\varepsilon - \varepsilon^*).
    \]
\end{remark}
%
%
%
\ifSubfilesClassLoaded{%
  \bibliography{v2-subfiles/manifolds-references}%
}{}
\end{document}